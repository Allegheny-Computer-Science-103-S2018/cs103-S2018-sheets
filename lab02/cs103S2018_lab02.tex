\documentclass[11pt]{article}

% NOTE: The "Edit" sections are changed for each assignment

% Edit these commands for each assignment

\newcommand{\assignmentduedate}{February 8}
\newcommand{\assignmentassignedate}{February 1}
\newcommand{\assignmentnumber}{Two}

\newcommand{\labyear}{2018}
\newcommand{\labday}{Thursday}
\newcommand{\labtime}{2:30 pm}

\newcommand{\assigneddate}{Assigned: \labday, \assignmentassignedate, \labyear{} at \labtime{}}
\newcommand{\duedate}{Due: \labday, \assignmentduedate, \labyear{} at \labtime{}}

% Edit these commands to give the name to the main program

\newcommand{\mainprogram}{\lstinline{answers.md}}
\newcommand{\mainprogramsource}{\lstinline{src/www/html/answers.md}}

% Edit this commands to describe key deliverables

\newcommand{\reflection}{\lstinline{writing/reflection.md}}

% Commands to describe key development tasks

% --> Running gatorgrader.sh
\newcommand{\gatorgraderstart}{\command{./gatorgrader.sh --start}}
\newcommand{\gatorgradercheck}{\command{./gatorgrader.sh --check}}

% Commands to describe key git tasks

% NOTE: Could be improved, problems due to nesting

\newcommand{\gitcommitfile}[1]{\command{git commit #1}}
\newcommand{\gitaddfile}[1]{\command{git add #1}}

\newcommand{\gitadd}{\command{git add}}
\newcommand{\gitcommit}{\command{git commit}}
\newcommand{\gitpush}{\command{git push}}
\newcommand{\gitpull}{\command{git pull}}

\newcommand{\gitcommitmainprogram}{\command{git commit src/www/html/answers.md -m "Your
descriptive commit message"}}

% Use this when displaying a new command

\newcommand{\command}[1]{``\lstinline{#1}''}
\newcommand{\program}[1]{\lstinline{#1}}
\newcommand{\url}[1]{\lstinline{#1}}
\newcommand{\channel}[1]{\lstinline{#1}}
\newcommand{\option}[1]{``{#1}''}
\newcommand{\step}[1]{``{#1}''}

\usepackage{pifont}
\newcommand{\checkmark}{\ding{51}}
\newcommand{\naughtmark}{\ding{55}}

\usepackage{listings}
\lstset{
  basicstyle=\small\ttfamily,
  columns=flexible,
  breaklines=true
}

\usepackage{fancyhdr}

\usepackage[margin=1in]{geometry}
\usepackage{fancyhdr}

\pagestyle{fancy}

\fancyhf{}
\rhead{Computer Science 103}
\lhead{Laboratory Assignment \assignmentnumber{}}
\rfoot{Page \thepage}
\lfoot{\duedate}

\usepackage{titlesec}
\titlespacing\section{0pt}{6pt plus 4pt minus 2pt}{4pt plus 2pt minus 2pt}

\newcommand{\labtitle}[1]
{
  \begin{center}
    \begin{center}
      \bf
      CMPSC 103\\Web Development\\
      Spring 2018\\
      \medskip
    \end{center}
    \bf
    #1
  \end{center}
}

\begin{document}

\thispagestyle{empty}

\labtitle{Laboratory \assignmentnumber{} \\ \assigneddate{} \\ \duedate{}}

\section*{Objectives}

To continue to practice using GitHub to access the files for a laboratory
assignment. Also, to learn more about using the Ubuntu operating system and
software development programs such as a ``terminal window'' and the ``GVim text
editor''. Next, you will learn how to write a Markdown document and convert it
to HTML, specifically investigating the application of a cascading style sheet
(CSS) to the generated HTML. Finally, you will learn more about running a web
server and using an automated grading tool to assess your completion of a web
development project.

\section*{Suggestions for Success}

\begin{itemize}
  \setlength{\itemsep}{0pt}

\item {\bf Use the laboratory computers}. The computers in this laboratory feature specialized software for completing
  this course's laboratory and practical assignments. If it is necessary for you to work on a different machine, be sure
  to regularly transfer your work to a laboratory machine so that you can check its correctness. If you cannot use a
  laboratory computer and you need help with the configuration of your own laptop, then please carefully explain its
  setup to a teaching assistant or the course instructor when you are asking questions.

\item {\bf Follow each step carefully}. Slowly read each sentence in the assignment sheet, making sure that you
  precisely follow each instruction. Take notes about each step that you attempt, recording your questions and ideas and
  the challenges that you faced. If you are stuck, then please tell a teaching assistant or instructor what assignment
  step you recently completed.

\item {\bf Regularly ask and answer questions}. Please log into Slack at the start of a laboratory or practical session
  and then join the appropriate channel. If you have a question about one of the steps in an assignment, then you can
  post it to the designated channel. Or, you can ask a student sitting next to you or talk with a teaching assistant or
  the course instructor.

\item {\bf Store your files in GitHub}. Starting with this laboratory assignment, you will be responsible for storing
  all of your files (e.g., JavaScript code and Markdown-based writing) in a Git repository using GitHub Classroom.
  Please verify that you have saved your source code in your Git repository by using \command{git status} to ensure that
  everything is updated. You can see if your assignment submission meets the established correctness requirements by
  using the provided checking tools on your local computer and in checking the commits in GitHub.

\item {\bf Keep all of your files}. Don't delete your programs, output files, and written reports after you submit them
  through GitHub; you will need them again when you study for the quizzes and examinations and work on the other
  laboratory, practical, and final project assignments.

\item {\bf Back up your files regularly}. All of your files are regularly backed-up to the servers in the Department of
  Computer Science and, if you commit your files regularly, stored on GitHub. However, you may want to use a flash
  drive, Google Drive, or your favorite backup method to keep an extra copy of your files on reserve. In the event of
  any type of system failure, you are responsible for ensuring that you have access to a recent backup copy of all your
  files.

\item {\bf Explore teamwork and technologies}. While certain aspects of the laboratory assignments will be challenging
  for you, each part is designed to give you the opportunity to learn both fundamental concepts in the field of computer
  science and explore advanced technologies that are commonly employed at a wide variety of companies. To explore and
  develop new ideas, you should regularly communicate with your team and/or the teaching assistants and tutors.

\item {\bf Hone your technical writing skills}. Computer science assignments require to you write technical
  documentation and descriptions of your experiences when completing each task. Take extra care to ensure that your
  writing is interesting and both grammatically and technically correct, remembering that computer scientists must
  effectively communicate and collaborate with their team members and the tutors, teaching assistants, and course
  instructor.

\item {\bf Review the Honor Code on the syllabus}. While you may discuss your assignments with others, copying source
  code or writing is a violation of Allegheny College's Honor Code.

\end{itemize}

\section*{Reading Assignment}

If you have not done so already, please read all of the relevant ``GitHub
Guides'', available at \url{https://guides.github.com/}, that explain how to use
many of the features that GitHub provides. In particular, make sure that you
have read guides such as ``Mastering Markdown'' and ``Documenting Your Projects
on GitHub''; each of them will help you to understand how to use both GitHub and
GitHub Classroom. To do well on this assignment, you should also read Chapters 1
and 2 in the course textbook, paying close attention to Sections 2.3 through
2.6, Figures 2.14 and 2.15, and Table 2.1. Students should also review the
Pandoc manual available at \url{https://github.com/jgm/pandoc} and the
description of Markdown styles at
\url{https://github.com/otsaloma/markdown-css}. Please see the course instructor
or a teaching assistant if you have questions these reading assignments.

\section*{Accessing the Laboratory Assignment on GitHub}

To access the laboratory assignment, you should go into the
\channel{\#announcements} channel in our Slack team and find the announcement
that provides a link for it. Copy this link and paste it into your web browser.
Now, you should accept the laboratory assignment and see that GitHub Classroom
created a new GitHub repository for you to access the assignment's starting
materials and to store the completed version of your assignment. Specifically,
to access your new GitHub repository for this assignment, please click the green
``Accept'' button and then click the link that is prefaced with the label ``Your
assignment has been created here''. If you accepted the assignment and correctly
followed these steps, you should have created a repository with a name like
``Allegheny-Computer-Science-103-Spring-2018/computer-science-103-spring-2018-lab-2-gkapfham''.
Unless you provide the course instructor with documentation of the extenuating
circumstances that you are facing, not accepting the assignment means that you
automatically receive a failing grade for it.

Before you move to the next step of this assignment, please make sure that you
read all of the content on the web site for your new GitHub repository, paying
close attention to the technical details about the commands that you will type
and the output that your program must produce. Now you are ready to download the
starting materials to your laboratory computer. Click the ``Clone or download''
button and, after ensuring that you have selected ``Clone with SSH'', please
copy this command to your clipboard. To enter into your course directory you
should now type \command{cd cs103S2018}. Now, by typing \command{git clone} in
your terminal and then pasting in the string that you copied from the GitHub
site you will download all of the code for this assignment. For instance, if the
course instructor ran the \command{git clone} command in the terminal, it would
look like:

\begin{lstlisting}
  git clone git@github.com:Allegheny-Computer-Science-103-S2018/computer-science-103-spring-2018-lab-2-gkapfham.git
\end{lstlisting}

After this command finishes, you can use \command{cd} to change into the new
directory. If you want to \step{go back} one directory from your current
location, then you can type the command \command{cd ..}. Please continue to use
the \command{cd} and \command{ls} commands to explore the files that you
automatically downloaded from GitHub. What files and directories do you see?
What do you think is their purpose? Please note that this assignment includes
many files, including two cascading style sheet (CSS) files. Spend some time
exploring, sharing your discoveries with a neighbor and a \mbox{teaching
assistant}.

\section*{Creating a Web Site Using Markdown and Style Sheets}

In order view the source code of a web site or a Markdown-based writing
assignment, you need a text editor. Since it is a powerful text editor known for
helping computer scientists ``edit text at the speed of thought'', in this class
we will often use the text editor called \command{gvim}. Today, you will view
and improve web site source code in {\tt gvim}. Making sure you are in the
directory for this assignment, you should type \command{gvim
src/www/html/answers.md} in your terminal. This should cause a new window, the
\program{gvim} text editor, to appear on your screen. Remember, since
\program{gvim} is a modal editor, you may notice that if you start typing,
nothing appears (unless you happen to hit certain letters such as \command{i},
\command{o}, \command{a}, and a few others). This is because you are not in
``insert mode.'' To get into insert mode, just type the letter \command{i}
(lower case). Once you do this, the window should show the word
\command{--INSERT--} in the lower left corner. Students who are having
difficulties learning how to use \command{gvim} should keep trying---but you can
also use the Atom text editor in this class.

This laboratory assignment invites you to implement a web site that displays
headers, subheaders, paragraphs, links, inline code segments, ``fenced'' code
segments, and a table. Please refer to your ``Markdown Syntax'' guide for
examples of each of these Markdown source code constructs. To continue the
assignment, you should edit the header at the top of the file to include your
name. Next, you should go below the header and, keeping a space between it and
your paragraph, write a paragraph that links to the Pandoc manual available at
\url{https://pandoc.org/MANUAL.html}, explaining one thing that you learned
about Pandoc. Your introductory paragraph should also acknowledge, this time
with a link to \url{https://github.com/otsaloma/markdown-css}, that you are
using externally provided cascading style sheets to influence the display of the
generated HTML.

You will notices that the \mainprogram{} also asks you to refer to source code
in both an ``inline'' and ``fenced'' fashion. This first type of formatting will
allow you to refer to the name of a file and see it appear in a distinct font.
For instance, a filename that is surrounded by backticks would be display as an
inline code segment. Alternatively, a fenced code segment starts with three
backticks, includes source code or program output, and then concludes with three
more backticks. You will use fenced code segments to display terminal commands
in your web site. Your \mainprogram{} should also include a table that reports
on your understanding of the content in Table 2.1 of your textbook.
Specifically, you should pick at least three error codes and explain them using
a Markdown table with the same format as Table 2.1. Again, please refer to the
``Markdown Syntax'' guide for useful examples and talk with the instructor if
you have questions about using these constructs.

Whenever you are finished typing text in \program{gvim}, press the ESC key
located in the upper left corner of the keyboard. This should remove the word
\command{--INSERT--} from the bottom of the screen and take you out of insert
mode. Use the \option{File/Save} menu to save your program. Alternatively, if
you would like to use the keyboard to save your file, you can press \command{:w}
when you are not in insert mode. Leaving the {\tt gvim} window open, go back to
your terminal window. Now, you are ready to use the tools that convert your
Markdown file into HTML and run a web server to make your web site available in
a browser. To convert your Markdown, make sure that you are in the ``home
directory'' of your GitHub repository and type one the following commands. Note
that these commands are different than what you used in a previous laboratory
assignment because they specify that output must adhere to the HTML5 standard
and apply a cascading style sheet in a single (i.e., ``self-contained'') file.
Please notice that the specifying a different \command{--css} argument will
cause the style of the generated HTML to vary. In summary, what does the web
site look like?

\vspace*{-.05in}

\begin{verbatim}
pandoc src/www/html/answers.md \
       --output=src/www/html/answers.html \
       --to=html5 \
       --css=src/www/css/github.css \
       --self-contained \
       --smart
\end{verbatim}

\vspace*{-.2in}

\begin{verbatim}
pandoc src/www/html/answers.md \
       --output=src/www/html/answers.html \
       --to=html5 \
       --css=src/www/css/tufte.css \
       --self-contained \
       --smart
\end{verbatim}

\vspace*{-.05in}

Now, you can run your web server by typing the command \command{serve
src/www/html 4250}. At this point, you can start your web browser and go to the
web site \url{http://localhost:4250/answers.html}. Does your new web site look
correct? If not, then continue to edit and convert it until the files are
correct. Once the web site's content is correct, please reflect on this process.
As you iteratively complete this assignment, you should regularly commit to your
GitHub repository, following the steps in the next section. Also, note that your
web server will require a dedicated terminal when it is running. Now, reflect on
the entire process. What step did you find to be the most challenging? Why? You
should write your reflections in a file, called \reflection{}, that also uses
the Markdown writing language. To complete this aspect of the assignment, you
should write one high-quality paragraph that reports on your experiences. Now,
verbally share your experiences with another class member and the course
instructor and at least one the teaching assistants!

\section*{Checking the Correctness of Your Web Site and Writing}

The Markdown file that contains your reflection must adhere to the standards
described in the Markdown Syntax Guide
\url{https://guides.github.com/features/mastering-markdown/}. Finally, your
\reflection{} file should adhere to the Markdown standards established by the
\step{Markdown linting} tool available at
\url{https://github.com/markdownlint/markdownlint/} and the writing standards
set by the \step{prose linting} tool from \url{http://proselint.com/}. Instead
of requiring you to manually check that your deliverables adhere to these
industry-accepted standards, the GatorGrader tool that you will use in this
laboratory assignment makes it easy for you to automatically check if your
submission meets the standards for correctness. For instance, GatorGrader will
check to ensure that \mainprogram{} has the required five subsections created
with the \command{##} code. It will also check to ensure that you create the
required number of fenced source code blocks.

To get started with the use of GatorGrader, type the command \gatorgraderstart{}
in your terminal window. Once this step completes you can type
\gatorgradercheck{}. If your work does not meet all of the assignment's
requirements, then you will see the following output in your terminal:
\command{Overall, are there any mistakes in the assignment? Yes}. If you do have
mistakes in your assignment, then you will need to review GatorGrader's output,
find the mistake, and try to fix it. Once your static web site displays
correctly, fulfilling at least some of the assignment's requirements, you should
transfer your files to GitHub using the \gitcommit{} and \gitpush{} commands.
For example, if you want to signal that the \mainprogramsource{} file has been
changed and is ready for transfer to GitHub you would first type
\gitcommitmainprogram{} in your terminal, followed by typing \gitpush{} and
checking to see that the transfer to GitHub is successful. If you notice that
transferring your code or writing to GitHub did not work, then please read the
messages in your terminal and try to determine why, asking the course instructor
for assistance, if necessary.

After the course instructor enables \step{continuous integration} with a system
called Travis CI, when you use the \gitpush{} command to transfer your source
code to your GitHub repository, Travis CI will initialize a \step{build} of your
assignment, checking to see if it meets all of the requirements. If both your
source code and writing meet all of the established requirements, then you will
see a green \checkmark{} in the listing of commits in GitHub after awhile. If
your submission does not meet the requirements, a red \naughtmark{} will appear
instead. The instructor will reduce a student's grade for this assignment if the
red \naughtmark{} appears on the last commit in GitHub immediately before the
assignment's due date. Yet, if the green \checkmark{} appears on the last commit
in your GitHub repository, then you satisfied all of the main checks, thereby
allowing the course instructor to evaluate other aspects of your source code and
writing, as further described in the \step{Evaluation} section of this
assignment sheet. Unless you provide the instructor with documentation of the
extenuating circumstances that you are facing, no late work will be considered
towards your grade for this laboratory assignment.

\section*{Summary of the Required Deliverables}

\noindent Students do not need to submit printed source code or technical writing for any assignment in this course.
Instead, this assignment invites you to submit, using GitHub, the following deliverables.

\begin{enumerate}

\setlength{\itemsep}{0in}

\item Stored in \reflection{}, a one-paragraph reflection on the commands that you typed in \command{gvim} and the
  terminal window. This Markdown-based document should explain the input, output, and behavior of each command and the
  challenges that you confronted when using it. For every challenge that you encountered, please explain your solution
  for it.

\item A properly formatted and correct version of \mainprogramsource{} that both meets all of the established
  requirements and produces the correct HTML and the desired static web site.

\end{enumerate}

\section*{Evaluation of Your Laboratory Assignment}

Using a report that the instructor shares with you through the commit log in GitHub, you will privately received a grade
on this assignment and feedback on your submitted deliverables. Your grade for the assignment will be a function of the
whether or not it was submitted in a timely fashion and if your program received a green \checkmark{} indicating that it
met all of the requirements. Other factors will also influence your final grade on the assignment. In addition to
studying the efficiency and effectiveness of your Markdown, the instructor will also evaluate the accuracy of both your
technical writing and the contents of your source code. If your submission receives a red \naughtmark{}, the instructor
will reduce your grade for the assignment while still considering the regularity with which you committed to your GitHub
repository and the overall quality of your partially completed work. Please see the instructor if you have questions
about the evaluation of this laboratory assignment.

\section*{Adhering to the Honor Code}

In adherence to the Honor Code, students should complete this assignment on an individual basis. While it is appropriate
for students in this class to have high-level conversations about the assignment, it is necessary to distinguish
carefully between the student who discusses the principles underlying a problem with others and the student who produces
assignments that are identical to, or merely variations on, someone else's work. Deliverables (e.g., HTML source code or
Markdown-based technical writing) that are nearly identical to the work of others will be taken as evidence of violating
the \mbox{Honor Code}. Please see the course instructor if you have questions about this policy.

\end{document}
