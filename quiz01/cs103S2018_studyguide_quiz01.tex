\documentclass[11pt]{article}

% NOTE: The "Edit" sections are changed for each assignment

% Edit these commands for each assignment

\newcommand{\assignmentduedate}{May 16}
\newcommand{\assignmentassignedate}{March 7}
\newcommand{\assignmentnumber}{One}

\newcommand{\labyear}{2018}
\newcommand{\assignedday}{Wednesday}
\newcommand{\dueday}{Friday}
\newcommand{\labtime}{1:30 pm}

\newcommand{\assigneddate}{Announced: \assignedday, \assignmentassignedate, \labyear{} at \labtime{}}
\newcommand{\duedate}{Quiz: \dueday, \assignmentduedate, \labyear{} at \labtime{}}

% Edit these commands to give the name to the main program

\newcommand{\mainprogram}{\lstinline{DisplayOutput}}
\newcommand{\mainprogramsource}{\lstinline{src/main/java/labone/DisplayOutput.java}}

% Edit this commands to describe key deliverables

\newcommand{\reflection}{\lstinline{writing/reflection.md}}

% Commands to describe key development tasks

% --> Running gatorgrader.sh
\newcommand{\gatorgraderstart}{\command{./gatorgrader.sh --start}}
\newcommand{\gatorgradercheck}{\command{./gatorgrader.sh --check}}

% --> Compiling and running program with gradle
\newcommand{\gradlebuild}{\command{gradle build}}
\newcommand{\gradlerun}{\command{gradle run}}

% Commands to describe key git tasks

% NOTE: Could be improved, problems due to nesting

\newcommand{\gitcommitfile}[1]{\command{git commit #1}}
\newcommand{\gitaddfile}[1]{\command{git add #1}}

\newcommand{\gitadd}{\command{git add}}
\newcommand{\gitcommit}{\command{git commit}}
\newcommand{\gitpush}{\command{git push}}
\newcommand{\gitpull}{\command{git pull}}

\newcommand{\gitcommitmainprogram}{\command{git commit src/main/java/labone/DisplayOutput.java -m "Your
descriptive commit message"}}

% Use this when displaying a new command

\newcommand{\command}[1]{``\lstinline{#1}''}
\newcommand{\program}[1]{\lstinline{#1}}
\newcommand{\url}[1]{\lstinline{#1}}
\newcommand{\channel}[1]{\lstinline{#1}}
\newcommand{\option}[1]{``{#1}''}
\newcommand{\step}[1]{``{#1}''}

\usepackage{pifont}
\newcommand{\checkmark}{\ding{51}}
\newcommand{\naughtmark}{\ding{55}}

\usepackage{listings}
\lstset{
  basicstyle=\small\ttfamily,
  columns=flexible,
  breaklines=true
}

\usepackage{fancyhdr}

\usepackage[margin=1in]{geometry}
\usepackage{fancyhdr}

\pagestyle{fancy}

\fancyhf{}
\rhead{Computer Science 103}
\lhead{Quiz \assignmentnumber{}}
\rfoot{Page \thepage}
\lfoot{\duedate}

\usepackage{titlesec}
\titlespacing\section{0pt}{6pt plus 4pt minus 2pt}{4pt plus 2pt minus 2pt}

\newcommand{\guidetitle}[1]
{
  \begin{center}
    \begin{center}
      \bf
      CMPSC 103\\Introduction to Computer Science II\\
      Spring 2018\\
      \medskip
    \end{center}
    \bf
    #1
  \end{center}
}

\begin{document}

\thispagestyle{empty}

\guidetitle{Quiz \assignmentnumber{} Study Guide \\ \assigneddate{} \\ \duedate{}}

\section*{Introduction}

\noindent
The quiz will be ``closed notes'' and ``closed book'' and it will cover the
following materials. Please review the ``Course Schedule'' on the Web site for
the course to see the content and slides that we have covered to this date.
Students may post questions about this material to our Slack team.

\begin{itemize}

  \itemsep 0in

  \item Chapters One through Five in FWD, all sections (see online schedule for
    complete details).

  \item Using the commands in the terminal window (e.g., \program{cd} and
    \program{ls}); creating and viewing web sites with Markdown, HTML, and CSS;
    knowledge of the commands for using \program{git} and GitHub.

  \item Your class notes and the discussion slides available from the course web
    site.

  \item Source code and writing for laboratory assignments 1--7 and practical
    assignments 1--6.

\end{itemize}

\noindent The quiz will be a mix of questions that have a form such as fill in
the blank, short answer, true/false, and completion. The emphasis will be on the
following topics:

\vspace*{-.05in}
\begin{itemize}

  \itemsep 0in

  \item Concepts in web development and the HTML and CSS languages (e.g., key
    definitions).

  \item Practical laboratory techniques (e.g., creating, debugging, and viewing
    web sites; effectively using files and directories; correctly using GitHub
    through the command-line {\tt git} program).

  \item Understanding Web sites (e.g., given a short, perhaps even one line,
    source code segment written in Markdown, HTML, and/or CSS, understand what
    it does and be able to precisely describe its resulting output as evident in
    the web server logs or in the web browser).

  \item Programming Markdown, HTML, and/or CSS statements and web sites, given a
    description of a site's appearance and behavior. Students should be
    completely comfortable with writing short source code statements that are in
    a nearly correct form. While your web site may contain small syntactic
    errors, it is not acceptable to ``make up'' features of the Markdown, HTML,
    and CSS programming language that do not currently exist. Given a source
    code segment you should also be able to sketch what a web site would look
    like in a web browser.

\end{itemize}

\noindent No partial credit will be given for questions that are true/false,
completion, or fill in the blank. Minimal partial credit may be awarded for the
questions that require a student to write a short answer. You are strongly
encouraged to write short, precise, and correct responses to all of the
questions. When you are taking the quiz, you should do so as a ``point
maximizer'' who first responds to the questions that you are most likely to
answer correctly for full points. Please keep the time limitation in mind as you
are absolutely required to submit the examination at the end of the class period
unless you have written permission for extra time from a member of the Learning
Commons. Students who do not submit their quiz on time will have their overall
point total reduced. Finally, students may only reschedule the quiz for a
different date or time if they are facing documented extenuating circumstances
that prevent them from attending the scheduled time slot. Please see the course
instructor if you have questions about any of these policies.

\section*{Reminder Concerning the Honor Code}

\noindent Students are required to fully adhere to the Honor Code during the
completion of this quiz. More details about the Allegheny College Honor Code are
provided on the syllabus. Students are strongly encouraged to carefully review
the full statement of the Honor Code before taking this quiz. If you do not
understand Allegheny College's Honor Code, please schedule a meeting with the
course instructor. The following is a review of Honor Code statement from the
course syllabus:

% \vspace*{-.1in}

\begin{quote}
The Academic Honor Program that governs the entire academic program at
Allegheny College is described in the Allegheny Academic Bulletin. The Honor
Program applies to all work that is submitted for academic credit or to meet
non-credit requirements for graduation at Allegheny College. This includes all
work assigned for this class (e.g., examinations, laboratory assignments, and
the final project). All students who have enrolled in the College will work
under the Honor Program. Each student who has matriculated at the College has
acknowledged the following pledge:
\end{quote}

\vspace*{-.1in}

\begin{quote}
  I hereby recognize and pledge to fulfill my responsibilities, as defined in
  the Honor Code, and to maintain the integrity of both myself and the College
  community as a whole.
\end{quote}

\section*{Detailed Review of Content}

The listing of topics in the following subsections is not exhaustive; rather,
it serves to illustrate the types of concepts that students should study as
they prepare for the quiz. Please see the course instructor during office hours
if you have questions about any of the content listed in this section.

\vspace*{-.1in}

\subsection*{Chapter One}

\begin{itemize}

  \itemsep 0in

  \item The similarities and differences between an intranet and the Internet
  \item The similarities and differences between static and dynamic web sites
  \item An Understanding of the client-server model and the ``request-response''
    loop
  \item How networks, routers, and Internet service providers enable a
    connection to a web site
  \item The different roles and skills associated with the field of web development
  \item A high-level understanding of the technologies used to create web sites
  \item The role that a version control system plays in the development of a web
    site

\end{itemize}

\vspace*{-.2in}
\subsection*{Chapter Two}

\begin{itemize}

  \itemsep 0in

  \item The goals, principles, and patterns of object-oriented design in the
    Java language
  \item An understanding of the principles known as abstraction, encapsulation, and modularity
  \item How to use inheritance hierarchies to create an ``is a'' relationship
    between Java classes
  \item The meaning and purpose of abstract classes and interfaces in the Java
    programming language
  \item How to create, catch, and handle exceptions thrown in a Java program

\end{itemize}

\vspace*{-.2in}
\subsection*{Chapter Three}

\begin{itemize}

  \itemsep 0in

  \item How to use arrays to store primitive and reference variables
  \item The algorithms for sorting arrays into ascending and descending order
  \item How to use psuedo random number generators in Java programs
  \item The similarities and differences between one- and two-dimensional arrays
  \item The meaning and purpose of techniques for cloning data structures

\end{itemize}

\vspace*{-.2in}
\subsection*{Chapter Four}

\begin{itemize}

  \itemsep 0in

  \item A strategy for timing the implementation of an algorithm in Java
  \item The challenges associated with experimentally studying an algorithm's
    performance
  \item A comprehensive understanding of how to use a doubling experiment to
    study efficiency
  \item The challenges associated with conduct an experimental evaluation of an
    algorithm
  \item The meaning and purpose of the terms ``basic operation'' ``psuedo code''
  \item An intuitive understanding of best-, worst-, and average-case analytical
    evaluations

\end{itemize}

\vspace*{-.2in}
\subsection*{Chapter Five}

\begin{itemize}

  \itemsep 0in

  \item How to use a recursive approach to solving a problem (e.g., the base
    and recursive cases)
  \item Knowledge of recursive algorithms for arithmetic computation (e.g., factorial
    and Fibonacci)
  \item How to determine and justify the worst-case time complexity for
    the recursive factorial method
  \item The similarities and differences between linear, binary, and multiple
    recursion
  \item An understanding of how different recursive strategies influence the
    efficiency of an algorithm

\end{itemize}

\end{document}

