\documentclass[11pt]{article}

% NOTE: The "Edit" sections are changed for each assignment

% Edit these commands for each assignment

\newcommand{\assignmentduedate}{February 19}
\newcommand{\assignmentassignedate}{February 16}
\newcommand{\assignmentnumber}{Three}

\newcommand{\labyear}{2018}
\newcommand{\labdueday}{Monday}
\newcommand{\labassignday}{Friday}
\newcommand{\labtime}{9:00 am}

\newcommand{\assigneddate}{Assigned: \labassignday, \assignmentassignedate, \labyear{} at \labtime{}}
\newcommand{\duedate}{Due: \labdueday, \assignmentduedate, \labyear{} at \labtime{}}

% Edit these commands to give the name to the main program

\newcommand{\mainprogram}{\lstinline{index.html}}
\newcommand{\mainprogramsource}{\lstinline{src/www/index.html}}

% Commands to describe key development tasks

% --> Running gatorgrader.sh
\newcommand{\gatorgraderstart}{\command{./gatorgrader.sh --start}}
\newcommand{\gatorgradercheck}{\command{./gatorgrader.sh --check}}

% --> Compiling and running program with gradle
\newcommand{\gradlebuild}{\command{gradle build}}
\newcommand{\gradlerun}{\command{gradle run}}

% Commands to describe key git tasks

% NOTE: Could be improved, problems due to nesting

\newcommand{\gitcommitfile}[1]{\command{git commit #1}}
\newcommand{\gitaddfile}[1]{\command{git add #1}}

\newcommand{\gitadd}{\command{git add}}
\newcommand{\gitcommit}{\command{git commit}}
\newcommand{\gitpush}{\command{git push}}
\newcommand{\gitpull}{\command{git pull}}

\newcommand{\gitcommitmainprogram}{\command{git commit src/www/index.html -m "Your
descriptive commit message"}}

% Use this when displaying a new command

\newcommand{\command}[1]{``\lstinline{#1}''}
\newcommand{\program}[1]{\lstinline{#1}}
\newcommand{\url}[1]{\lstinline{#1}}
\newcommand{\channel}[1]{\lstinline{#1}}
\newcommand{\option}[1]{``{#1}''}
\newcommand{\step}[1]{``{#1}''}

\usepackage{upquote}

\usepackage{pifont}
\newcommand{\checkmark}{\ding{51}}
\newcommand{\naughtmark}{\ding{55}}

\usepackage{listings}
\lstset{
  basicstyle=\small\ttfamily,
  columns=flexible,
  breaklines=true
}

\usepackage{fancyhdr}

\usepackage[margin=1in]{geometry}
\usepackage{fancyhdr}

\pagestyle{fancy}

\fancyhf{}
\rhead{Computer Science 103}
\lhead{Laboratory Assignment \assignmentnumber{}}
\rfoot{Page \thepage}
\lfoot{\duedate}

\usepackage{titlesec}
\titlespacing\section{0pt}{6pt plus 4pt minus 2pt}{4pt plus 2pt minus 2pt}

\newcommand{\labtitle}[1]
{
  \begin{center}
    \begin{center}
      \bf
      CMPSC 103\\Web Development\\
      Spring 2018\\
      \medskip
    \end{center}
    \bf
    #1
  \end{center}
}

\begin{document}

\thispagestyle{empty}

\labtitle{Practical \assignmentnumber{} \\ \assigneddate{} \\ \duedate{}}

\section*{Objectives}

To learn how to use GitHub to access the files for a practical assignment.
Additionally, to keep learning how to use the Ubuntu operating system and
programs such as a ``terminal window'' and the ``GVim text editor''. You will
also continue to practice using Slack to support communication with the teaching
assistants and the course instructor. Next, you will practice creating ordered
and unordered lists in HTML and making links to HTML files. You will also
practice running a web server and using the grading tool to assess your progress
towards correctly completing the project.

\section*{Suggestions for Success}

\begin{itemize}
  \setlength{\itemsep}{0pt}

\item {\bf Use the laboratory computers}. The computers in this laboratory feature specialized software for completing
  this course's laboratory and practical assignments. If it is necessary for you to work on a different machine, be sure
  to regularly transfer your work to a laboratory machine so that you can check its correctness. If you cannot use a
  laboratory computer and you need help with the configuration of your own laptop, then please carefully explain its
  setup to a teaching assistant or the course instructor when you are asking questions.

\item {\bf Follow each step carefully}. Slowly read each sentence in the assignment sheet, making sure that you
  precisely follow each instruction. Take notes about each step that you attempt, recording your questions and ideas and
  the challenges that you faced. If you are stuck, then please tell a teaching assistant or instructor what assignment
  step you recently completed.

\item {\bf Regularly ask and answer questions}. Please log into Slack at the start of a laboratory or practical session
  and then join the appropriate channel. If you have a question about one of the steps in an assignment, then you can
  post it to the designated channel. Or, you can ask a student sitting next to you or talk with a teaching assistant or
  the course instructor.

\item {\bf Store your files in GitHub}. Continuing with this laboratory assignment, you will be responsible for storing
  all of your files (e.g., JavaScript code and Markdown-based writing) in a Git repository using GitHub Classroom.
  Please verify that you have saved your source code in your Git repository by using \command{git status} to ensure that
  everything is updated. You can see if your assignment submission meets the established correctness requirements by
  using the provided checking tools on your local computer and in checking the commits in GitHub.

\item {\bf Keep all of your files}. Don't delete your programs, output files, and written reports after you submit them
  through GitHub; you will need them again when you study for the quizzes and examinations and work on the other
  laboratory, practical, and final project assignments.

\item {\bf Back up your files regularly}. All of your files are regularly backed-up to the servers in the Department of
  Computer Science and, if you commit your files regularly, stored on GitHub. However, you may want to use a flash
  drive, Google Drive, or your favorite backup method to keep an extra copy of your files on reserve. In the event of
  any type of system failure, you are responsible for ensuring that you have access to a recent backup copy of all your
  files.

\item {\bf Explore teamwork and technologies}. While certain aspects of the laboratory assignments will be challenging
  for you, each part is designed to give you the opportunity to learn both fundamental concepts in the field of computer
  science and explore advanced technologies that are commonly employed at a wide variety of companies. To explore and
  develop new ideas, you should regularly communicate with your team and/or the teaching assistants and tutors.

\item {\bf Hone your technical writing skills}. Computer science assignments require to you write technical
  documentation and descriptions of your experiences when completing each task. Take extra care to ensure that your
  writing is interesting and both grammatically and technically correct, remembering that computer scientists must
  effectively communicate and collaborate with their team members and the tutors, teaching assistants, and course
  instructor.

\item {\bf Review the Honor Code on the syllabus}. While you may discuss your assignments with others, copying source
  code or writing is a violation of Allegheny College's Honor Code.

\end{itemize}

\section*{Reading Assignment}

If you have not done so already, please read all of the relevant ``GitHub
Guides'', available at \url{https://guides.github.com/}, that explain how to use
many of the features that GitHub provides. In particular, please make sure that
you have read guides such as ``Mastering Markdown'' and ``Documenting Your
Projects on GitHub''; each of them will help you to understand how to use both
GitHub and GitHub Classroom. To do well on this assignment, you should also read
Chapters 1 through 3 in the course textbook, also paying close attention to
Section 3.5 and Listing 3.2. Please see the instructor or one of the teaching
assistants if you have questions these reading assignments.

\section*{Accessing the Practical Assignment on GitHub}

To access the practical assignment, you should go into the \channel{\#announcements} channel in our Slack team and find
the announcement that provides a link for it. Copy this link and paste it into your web browser. Now, you should accept
the practical assignment and see that GitHub Classroom created a new GitHub repository for you to access the
assignment's starting materials and to store the completed version of your assignment. Specifically, to access your new
repository for this assignment, please click the green ``Accept'' button and then click the link that is prefaced with
the label ``Your assignment has been created here''. If you accepted the assignment and correctly followed these steps,
you should have created a repository with a name like
``Allegheny-Computer-Science-103-Spring-2018/computer-science-103-spring-2018-practical-3-gkapfham''. Unless you provide
the course instructor with documentation of the extenuating circumstances that you are facing, not accepting the
assignment means that you automatically receive a failing grade for it.

Before you move to the next step of this assignment, please make sure that you read all of the content on the web site
for your new GitHub repository, paying close attention to the technical details about the commands that you will type
and the output that your program must produce. Now you are ready to download the starting materials to your laboratory
computer. Click the ``Clone or download'' button and, after ensuring that you have selected ``Clone with SSH'', please
copy this command to your clipboard. To enter your course directory you should now type \command{cd cs103S2018}. Next,
you can type the command \command{ls} and see that there are files or directories inside of this directory. By typing
\command{git clone} in your terminal and then pasting in the string that you copied from the GitHub site you will
download all of the code for this assignment. For instance, if the course instructor ran the \command{git clone} command
in the terminal, it would look like:

\begin{lstlisting}
  git clone git@github.com:Allegheny-Computer-Science-103-S2018/computer-science-103-spring-2018-practical-3-gkapfham.git
\end{lstlisting}

After this command finishes, you can use \command{cd} to change into the new directory. What files and directories do
you see? What do you think is their purpose? Please ask questions if you have any!

\begin{figure}[t]
  \centering

  \begin{verbatim}
      <ul>
        <li><a href="about.html" target="_blank">About</a></li>
        <li><a href="software.html" target="_blank">Software</a></li>
      </ul>
  \end{verbatim}

  \vspace*{-.25in}
  \begin{center}
    (a) Create an unordered lists that has links that open in a new browser tab.
  \end{center}

  \begin{verbatim}
      <ul>
        <li><a href="about.html">About</a></li>
        <li><a href="software.html">Software</a></li>
      </ul>
  \end{verbatim}

  \vspace*{-.25in}
  \begin{center}
    (a) Create an unordered lists that has links that do not open in a new browser tab.
  \end{center}

  \caption{Segments of Correct HTML Source Code.}~\label{fig:correct}

  \vspace*{-2em}

\end{figure}

\section*{Creating Ordered and Unordered Lists of Links}

This practical assignment invites you to create ordered and unordered lists that
link to other local files in a web site. Before you start to create these lists
you should first include the cascading style sheet (CSS) files that you used in
the previous laboratory assignment. Now, you are ready to create the lists that
will link to the \program{about.html} and \program{software.html} files.
Specifically, you will create both ordered and unordered lists that have links
that both do and do not open in a new browser tab. For instance,
Figure~\ref{fig:correct} furnishes an example of HTML source code segments that
will create unordered lists with links that both do and do not open in different
tabs. Can you add this source code to the correct location in the
\mainprogramsource{} file? You will also need to ensure that the default
document contains an ordered list that links to the same HTML files in the same
fashion. Finally, don't forget to use the \command{<footer>} tag to add at least
two emoji and your name to the bottom of the web site. In summary, you can learn
what code you need to add to the \mainprogram{} file, by reading all of the
\command{TODO} comments in the source code and talking with the instructor.

You can now run your web server by typing the command \command{serve src/www
4250}. At this point, you can start your web browser and go to the web site
\url{http://localhost:4250/}. Does your new web site look correct? Can you see
the lists? Do the links work? Do the emoji correctly appear in the footer? If
not, then continue to edit and convert it until the files are correct. Since
this is our third practical assignment and you are still learning how to use the
appropriate hardware and software, don't become frustrated if you make a
mistake. Instead, use your mistakes as an opportunity for learning both about
the necessary web development technology and the background and expertise of the
other students in the class, the teaching assistants, and the course instructor.

\section*{Checking the Correctness of Your Web Site and Writing}

The Markdown file that contains your reflection must adhere to the standards
described in the Markdown Syntax Guide
\url{https://guides.github.com/features/mastering-markdown/}. Finally, your
\mainprogram{} file should adhere to the Markdown standards established by the
\step{Markdown linting} tool available at
\url{https://github.com/markdownlint/markdownlint/} and the writing standards
set by the \step{prose linting} tool from \url{http://proselint.com/}. Instead
of requiring you to manually check that your deliverables adhere to these
industry-accepted standards, the GatorGrader tool that you will use in this
practical assignment makes it easy for you to automatically check if your
submission meets the standards for correctness. For instance, GatorGrader will
check to ensure that \mainprogram{} has the required number of unordered lists
created with the \command{<ul>} code.

To get started with the use of GatorGrader, type the command \gatorgraderstart{}
in your terminal window. Once this step completes you can type
\gatorgradercheck{}. If your work does not meet all of the assignment's
requirements, then you will see the following output in your terminal:
\command{Overall, are there any mistakes in the assignment? Yes}. If you do have
mistakes in your assignment, then you will need to review GatorGrader's output,
find the mistake, and try to fix it. Once your static web site displays
correctly, fulfilling at least some of the assignment's requirements, you should
transfer your files to GitHub using the \gitcommit{} and \gitpush{} commands.
For example, if you want to signal that the \mainprogramsource{} file has been
changed and is ready for transfer to GitHub you would first type
\gitcommitmainprogram{} in your terminal, followed by typing \gitpush{} and
checking to see that the transfer to GitHub is successful. If you notice that
transferring your code or writing to GitHub did not work correctly, then please
read the messages in your terminal and try to determine why, asking a teaching
assistant or the course instructor for assistance, if necessary.

After the course instructor enables \step{continuous integration} with a system
called Travis CI, when you use the \gitpush{} command to transfer your source
code to your GitHub repository, Travis CI will initialize a \step{build} of your
assignment, checking to see if it meets all of the requirements. If both your
source code and writing meet all of the established requirements, then you will
see a green \checkmark{} in the listing of commits in GitHub after awhile. If
your submission does not meet the requirements, a red \naughtmark{} will appear
instead. The instructor will reduce a student's grade for this assignment if the
red \naughtmark{} appears on the last commit in GitHub immediately before the
assignment's due date. Yet, if the green \checkmark{} appears on the last commit
in your GitHub repository, then you satisfied all of the main checks, thereby
allowing the course instructor to evaluate other aspects of your source code and
writing, as further described in the \step{Evaluation} section of this
assignment sheet. Unless you provide the instructor with documentation of the
extenuating circumstances that you are facing, no late work will be considered
towards your grade for this practical assignment.

\section*{Summary of the Required Deliverables}

\noindent Students do not need to submit printed source code or technical writing for any assignment in this course.
Instead, this assignment invites you to submit, using GitHub, the following deliverables.

\begin{enumerate}

\setlength{\itemsep}{0in}

\item A properly formatted and correct version of \mainprogramsource{} that both meets all of the established
  requirements and produces the correct HTML and the desired static web site.

\end{enumerate}

\section*{Evaluation of Your Practical Assignment}

Using a report that the instructor shares with you through the commit log in GitHub, you will privately received a grade
on this assignment and feedback on your submitted deliverables. Your grade for the assignment will be a function of the
whether or not it was submitted in a timely fashion and if your program received a green \checkmark{} indicating that it
met all of the requirements. Other factors will also influence your final grade on the assignment. In addition to
studying the efficiency and effectiveness of your Markdown source code, the instructor will also evaluate the accuracy of
both your writing and the constructs in your source code. If your submission receives a red \naughtmark{}, the
instructor will reduce your grade for the assignment while still considering the regularity with which you committed to
your GitHub repository and the overall quality of your partially completed work. Please see the instructor if you have
questions about the evaluation of this practical assignment.

% \section*{Adhering to the Honor Code}

% In adherence to the Honor Code, students should complete this assignment on an individual basis. While it is appropriate
% for students in this class to have high-level conversations about the assignment, it is necessary to distinguish
% carefully between the student who discusses the principles underlying a problem with others and the student who produces
% assignments that are identical to, or merely variations on, someone else's work. Deliverables (e.g., HTML source code or
% Markdown-based technical writing) that are nearly identical to the work of others will be taken as evidence of violating
% the \mbox{Honor Code}. Please see the course instructor if you have questions about this policy.

\end{document}
