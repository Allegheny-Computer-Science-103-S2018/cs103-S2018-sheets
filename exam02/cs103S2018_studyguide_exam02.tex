\documentclass[11pt]{article}

% NOTE: The "Edit" sections are changed for each assignment

% Edit these commands for each assignment

\newcommand{\assignmentduedate}{May 3}
\newcommand{\assignmentassignedate}{April 30}
\newcommand{\assignmentnumber}{Two}

\newcommand{\labyear}{2018}
\newcommand{\assignedday}{Monday}
\newcommand{\dueday}{Thursday}
\newcommand{\labtime}{9:00 am}

\newcommand{\assigneddate}{Announced: \assignedday, \assignmentassignedate, \labyear{} at \labtime{}}
\newcommand{\duedate}{Examination: \dueday, \assignmentduedate, \labyear{} at \labtime{}}

% Edit this commands to describe key deliverables

\newcommand{\reflection}{\lstinline{writing/reflection.md}}

% Commands to describe key development tasks

% --> Running gatorgrader.sh
\newcommand{\gatorgraderstart}{\command{./gatorgrader.sh --start}}
\newcommand{\gatorgradercheck}{\command{./gatorgrader.sh --check}}

% --> Compiling and running program with gradle
\newcommand{\gradlebuild}{\command{gradle build}}
\newcommand{\gradlerun}{\command{gradle run}}

% Commands to describe key git tasks

% NOTE: Could be improved, problems due to nesting

\newcommand{\gitcommitfile}[1]{\command{git commit #1}}
\newcommand{\gitaddfile}[1]{\command{git add #1}}

\newcommand{\gitadd}{\command{git add}}
\newcommand{\gitcommit}{\command{git commit}}
\newcommand{\gitpush}{\command{git push}}
\newcommand{\gitpull}{\command{git pull}}

\newcommand{\gitcommitmainprogram}{\command{git commit src/main/java/labone/DisplayOutput.java -m "Your
descriptive commit message"}}

% Use this when displaying a new command

\newcommand{\command}[1]{``\lstinline{#1}''}
\newcommand{\program}[1]{\lstinline{#1}}
\newcommand{\url}[1]{\lstinline{#1}}
\newcommand{\channel}[1]{\lstinline{#1}}
\newcommand{\option}[1]{``{#1}''}
\newcommand{\step}[1]{``{#1}''}

\usepackage{pifont}
\newcommand{\checkmark}{\ding{51}}
\newcommand{\naughtmark}{\ding{55}}

\usepackage{listings}
\lstset{
  basicstyle=\small\ttfamily,
  columns=flexible,
  breaklines=true
}

\usepackage{fancyhdr}

\usepackage[margin=1in]{geometry}
\usepackage{fancyhdr}

\pagestyle{fancy}

\fancyhf{}
\rhead{Computer Science 103}
\lhead{Examination \assignmentnumber{}}
\rfoot{Page \thepage}
\lfoot{\duedate}

\usepackage{titlesec}
\titlespacing\section{0pt}{6pt plus 4pt minus 2pt}{4pt plus 2pt minus 2pt}

\newcommand{\guidetitle}[1]
{
  \begin{center}
    \begin{center}
      \bf
      CMPSC 103\\Introduction to Web Development\\
      Spring 2018\\
      \medskip
    \end{center}
    \bf
    #1
  \end{center}
}

\begin{document}

\thispagestyle{empty}

\guidetitle{Examination \assignmentnumber{} Study Guide \\ \assigneddate{} \\ \duedate{}}

\section*{Introduction}

\noindent
The exam will be ``closed notes'' and ``closed book'' and it will cover the
following materials. Please review the ``Course Schedule'' on the Web site for
the course to see the content and slides that we have covered to this date.
Students may post questions about this material to our Slack team.

\begin{itemize}

  \itemsep 0in

  \item Chapters 1 through 9 and 20 in FWD, the specified sections (see online
    schedule for details).

  \item Using the commands in the terminal window (e.g., \program{cd} and
    \program{ls}); creating and viewing web sites with Markdown, HTML, and CSS;
    knowledge of the commands for using \program{git} and GitHub.

  \item All of your class notes and the discussion slides available from the
    course web site.

  \item Source code and writing for laboratory assignments 1--9 and practical
    assignments 1--8.

\end{itemize}

\noindent The exam will be a mix of questions that have a form such as fill in
the blank, short answer, true/false, and/or completion. The emphasis will be on
the following key topics:

\vspace*{-.05in}
\begin{itemize}

  \itemsep 0in

  \item Web development concepts in Markdown, HTML, CSS, and JavaScript (e.g.,
    key definitions).

  \item Practical laboratory techniques (e.g., creating, debugging, and viewing
    web sites; effectively using files and directories; correctly using GitHub
    through the command-line {\tt git} program).

  \item Understanding and fixing Web sites (e.g., given a short, perhaps even
    one line, source code segment written in Markdown, HTML, and/or CSS,
    understand what it does and be able to precisely describe its resulting
    output as evident in the web server logs or in the web browser).

  \item Programming Markdown, HTML, JavaScript and/or CSS in a web site, given a
    description of a site's appearance and behavior. Students should be
    completely comfortable with writing short source code statements that are in
    a nearly correct form. While your web site may contain small syntactic
    errors, it is not acceptable to ``make up'' features of the Markdown, HTML,
    and CSS programming language that do not currently exist. Given a source
    code segment you should also be able to sketch what a web site would look
    like in a web browser.

\end{itemize}

\noindent No partial credit will be given for questions that are true/false,
completion, or fill in the blank. Minimal partial credit may be awarded for the
questions that require a student to write a short answer. You are strongly
encouraged to write short, precise, and correct responses to all of the
questions. When you are taking the exam, you should do so as a ``point
maximizer'' who first responds to the questions that you are most likely to
answer correctly for full points. Please keep the time limitation in mind as you
are absolutely required to submit the examination at the end of the class period
unless you have written permission for extra time from a member of the Learning
Commons. Students who do not submit their exam on time will have their overall
point total reduced. Finally, students may only reschedule the exam for a
different date or time if they are facing documented extenuating circumstances
that prevent them from attending the scheduled time slot. Please see the course
instructor if you have questions about any of these policies.

\section*{Reminder Concerning the Honor Code}

\noindent Students are required to fully adhere to the Honor Code during the
completion of this exam. More details about the Allegheny College Honor Code are
provided on the syllabus. Students are strongly encouraged to carefully review
the full statement of the Honor Code before taking this exam. If you do not
understand Allegheny College's Honor Code, then please schedule a meeting with
the course instructor. The following is a review of the Honor Code statement
from the course syllabus:

\begin{quote}
The Academic Honor Program that governs the entire academic program at
Allegheny College is described in the Allegheny Academic Bulletin. The Honor
Program applies to all work that is submitted for academic credit or to meet
non-credit requirements for graduation at Allegheny College. This includes all
work assigned for this class (e.g., examinations, laboratory assignments, and
the final project). All students who have enrolled in the College will work
under the Honor Program. Each student who has matriculated at the College has
acknowledged the following pledge:
\end{quote}

\vspace*{-.15in}

\begin{quote}
  I hereby recognize and pledge to fulfill my responsibilities, as defined in
  the Honor Code, and to maintain the integrity of both myself and the College
  community as a whole.
\end{quote}

\section*{Detailed Review of Content}

The listing of topics in the following subsections is not exhaustive; rather, it
serves to illustrate the types of concepts that students should study as they
prepare for the examination. It is important to note that, by design, there is a
significant overlap in technical topics between this examination and the prior
mastery quiz. Please see the course instructor or during office hours or a class
session if you have questions about any of the content listed in the following
subsections.

\vspace*{-.15in}

\subsection*{Chapter One}

\begin{itemize}

  \itemsep 0in

  \item The similarities and differences between an intranet and the Internet
  \item The similarities and differences between static and dynamic web sites
  \item An understanding of the client-server model and the ``request-response''
    loop
  \item How networks, routers, and Internet service providers enable a
    connection to a web site
  \item The different roles and skills associated with the field of web development
  \item A high-level understanding of the technologies used to create web sites
  \item The role that a version control system plays in the development of a web
    site

\end{itemize}

\vspace*{-.2in}
\subsection*{Chapter Two}

\begin{itemize}

  \itemsep 0in

  \item An intuitive understanding of the Internet protocols supporting a
    connection to a web site
  \item How the Domain Name System supports the registration of and the access
    to a web site
  \item The meaning, purpose, and structure of a uniform resource locator for a
    web site
  \item The role that web server ports play in furnishing access to a web site
  \item Details concerning the hypertext transfer protocol (HTTP) and its
    common response codes
  \item A detailed understanding of the steps that a browser takes to parse HTML
    and handle requests
  \item Knowledge of the web server's activities and the terminal-based output
    of a web server
  \item The motivation for and ways in which both web browsers and servers employ
    caching
  \item A basic understanding of the various ``application stacks'' employed
    by different web servers

\end{itemize}

\vspace*{-.2in}
\subsection*{Chapter Three}

\begin{itemize}

  \itemsep 0in

  \item The term ``markup language'' and a historical understand of the
    hypertext markup language
  \item An understanding of the web-related terms ``tag'', ``markup'', ``data'',
    and ``meta data''
  \item Knowledge of the terms ``syntax'' and ``semantics'' and how they
    connect to web development
  \item Familiarity with the canonical structure of an HTML file that
    uses cascading style sheets
  \item An understanding of commonly used tags in an HTML file (e.g., \program{<head>},
    \program{<p>}, and \program{<div>})
  \item Given the directory tree for a web site, the ability to explain
    different relative references
  \item Strategies for include accessible images and international character
    entities in web sites
  \item The benefits associated with using semantic markup in the fifth version
    of the HTML
  \item The ability to use semantic markup tags to create, for instance, figures
    and figure captions
  \item How to appropriately use tools to support the development and
    debugging of web sites

\end{itemize}

\vspace*{-.2in}
\subsection*{Chapter Four}

\begin{itemize}

  \itemsep 0in

  \item The benefits and potential challenges of using cascading style sheets to
    style a web site
  \item The relationship between the CSS and the HTML source code in a web site
  \item The syntax and semantics of source code for declaring and applying CSS
    styles
  \item Knowledge of common CSS properties and the way in which they influence
    a page's display
  \item The rules that govern how CSS styles interact (e.g., inheritance,
    specificity, and location)
  \item The CSS box model and the use of margins and padding to influence a web
    page's formatting
  \item The way to use the ``TRBL'' shortcut to remind you how to control the
    borders of elements
  \item The ways to install, configure, and style fonts downloaded from Google's
    font archive
  \item An understanding of font sizes and the units of measurement used to
    specify their size
  \item The generic font families and how they connect the fonts provided by
    Google's font archive
  \item How use use both percentages and ``em units'' to specify the size of a
    font on a web site
  \item How to appropriately use tools to support the development and
    debugging of CSS files

\end{itemize}

\vspace*{-.2in}
\subsection*{Chapter Five}

\begin{itemize}

  \itemsep 0in

  \item The way in which HTML tables support the organization of data into rows
    and columns
  \item The meaning and behavior of the HTML tags supporting the creation of
    tables
  \item The similarities and differences between table headers, rows, and data
    elements
  \item The benefits and drawbacks associated with using tables for web page
    layout
  \item How to style HTML tables with CSS that changes, for instance, table
    borders and backgrounds
  \item An understanding of the CSS source code needed to create tables that
    have a ``zebra effect''
  \item The way to create interactive tables through the use of CSS-based
    properties for hovering
  \item The meaning and behavior of the HTML tags needed to create a form in a
    web page
  \item The similarities and differences in the appearance and behavior of form
    control elements
  \item Strategies for supporting forms in a static web site that does not have
    access to a database
  \item The challenges associated with creating date and time controls in a web
    page's form
  \item The motivation for and benefits of creating tables and forms that
    support accessibility
  \item How to appropriately use tools to support the development and
    debugging of tables and forms

\end{itemize}

\vspace*{-.2in}
\subsection*{Chapter Six}

\begin{itemize}

  \itemsep 0in

  \item The similarities and differences (and trade-offs) between raster images
    and vector images

  \item The different models (e.g., RGB and HSB) that are available for
    describing colors in web sites

  \item Meaning of terms like opacity and gradient and knowledge of how
    they connect to web sites

  \item The specific CSS source code needed to programming gradients for regions
    in web sites

  \item A basic knowledge of the relationships between colors (e.g.,
    complementary and analogous colors) and an understanding of tools that make
    color schemes (e.g., \program{http://paletton.com})

  \item The trade-offs (in, for instance, network bandwidth consumption or
    perceived image quality) associated with resizing images using either a
    conversion tool or HTML and CSS source code

  \item The meaning ``display resolution'' and its connection to the monitor
    size that displays a site

  \item An understanding of the file formats (and extensions) used to store
    images used on a web site

  \item The knowledge of the similarities and differences between lossy and
    lossless image compression

  \item An understanding of now image bit-depth relates the quality of an image
    displayed on a site

  \item Knowledge of the trade-offs associated with methods for organizing web
    media into a gallery

\end{itemize}

\vspace*{-.2in}
\subsection*{Chapter Seven}

\begin{itemize}

  \itemsep 0in

  \item An understanding of the term ``normal flow'' and how it influences the
    layout of web elements

  \item The ability to classify different HTML tags as being either block-level
    or inline elements

  \item Knowledge of both replaced inline and non-replaced inline elements and
    how they are placed

  \item Understanding of the similarities and differences associated with
    relative and absolute position

  \item The ability to write, understand, and debug CSS source code that
    positions HTML elements

  \item The knowledge of the reasons why it may be advantageous to put elements
    in a fixed position

  \item How to float HTML elements in either the entire web site or in a
    specific container

  \item The ability to understand and debug CSS source code that overlays and
    hides HTML elements

  \item A high-level understanding of the reasons why modern web sites need
    multicolumn layouts

  \item The similarities and differences between the terms ``fixed layout'' and
    ``fluid layout''

  \item Knowledge of the problems associated with viewing a web site that has a
    fixed or fluid layout

  \item Understanding the term ``viewport width'' and know why it is critical in
    web development

  \item The viewport widths for browsers that would run on desktop, laptop, and
    mobile devices

  \item Knowledge of the key principles and trade-offs associated with
    responsive web design

  \item An understanding of how media queries support the implementation of
    mobile-ready sites

  \item Examples of CSS frameworks (e.g., ``Twitter Bootstrap'') that support
    responsive web design

  \item Examples of the key features provided by CSS frameworks (e.g.,
    grid-based content layout)

\end{itemize}

\vspace*{-.2in}
\subsection*{Chapter Eight}

\begin{itemize}

  % \itemsep 0in

  \item The history of JavaScript and how it is similar to and different from
    other languages

  \item Knowledge of the different places at which a programmer can embed
    JavaScript source code

  \item The challenges and solutions associated with handling browsers that do
    not support JavaScript

  \item The variable declaration methods and data types that are available for a
    JavaScript program

  \item The similarities and differences between primitive and reference data
    types in JavaScript

  \item An understanding of the inputs, outputs, and behavior of the JavaScript
    methods for output

  \item How the JavaScript programming language supports the implementation of
    conditional logic

  \item What is means if a variable in JavaScript is described as being either
    ``truthy'' or ``falsy''.

  \item The JavaScript constructs (e.g., {\tt while} loops) available for the
    support of iterative computing

  \item Understanding of a strategy for picking the most suitable iteration
    construct for a web site

  \item How to create, manipulate, and index arrays in the JavaScript
    programming language

  \item The ways in which JavaScript supports dynamically typed object-oriented programming

  \item How to declare, invoke, and test a function in the JavaScript
    programming language

  \item Understanding of the term ``scope'' and how it influences the use of
    variables in JavaScript

  \item Detailed knowledge of the extended example's use of JavaScript to create
    a country gallery

\end{itemize}

\vspace*{-.2in}
\subsection*{Chapter Nine}

\begin{itemize}

  % \itemsep 0in

  \item How the document object model (DOM) represents and supports the
    change of a web site

  \item The way in which DOM methods support the selection and modification of
    HTML elements

  \item How the JavaScript language supports the change of an both element's
    style and content

  \item Knowledge of an appropriate ordering that first loads and then modifies
    the DOM tree

  \item The meaning of the term ``event-driven programming'' and how it connects
    to JavaScript

  \item An understanding of the different events triggered by web site
    interaction (e.g., {\tt click})

  \item The ways in which JavaScript supports the verification of the content
    input into a form

\end{itemize}

\vspace*{-.2in}
\subsection*{Chapter Twenty}

\begin{itemize}

  \item Knowledge of the popularity and benefits of JavaScript and its
    frameworks (e.g., Vue.js)

  \item The ways in which JavaScript frameworks support the implementation of
    interactive web sites

  \item The benefits of using frameworks like AngularJS to create single-page
    applications (SPAs)

  \item How JavaScript frameworks adopt the model-view-controller model for
    web programming

  \item How JavaScript frameworks like Ionic and React-Native support
    cross-platform development

\end{itemize}

\end{document}
