\documentclass[11pt]{article}

% NOTE: The "Edit" sections are changed for each assignment

% Edit these commands for each assignment

\newcommand{\assignmentduedate}{February 22}
\newcommand{\assignmentassignedate}{February 15}
\newcommand{\assignmentnumber}{Four}

\newcommand{\labyear}{2018}
\newcommand{\labday}{Thursday}
\newcommand{\labtime}{2:30 pm}

\newcommand{\assigneddate}{Assigned: \labday, \assignmentassignedate, \labyear{} at \labtime{}}
\newcommand{\duedate}{Due: \labday, \assignmentduedate, \labyear{} at \labtime{}}

% Edit these commands to give the name to the main program

\newcommand{\mainprogram}{\lstinline{index.html}}
\newcommand{\mainprogramsource}{\lstinline{src/www/index.html}}

% Edit this commands to describe key deliverables

\newcommand{\reflection}{\lstinline{writing/reflection.md}}

% Commands to describe key development tasks

% --> Running gatorgrader.sh
\newcommand{\gatorgraderstart}{\command{./gatorgrader.sh --start}}
\newcommand{\gatorgradercheck}{\command{./gatorgrader.sh --check}}

% Commands to describe key git tasks

% NOTE: Could be improved, problems due to nesting

\newcommand{\gitcommitfile}[1]{\command{git commit #1}}
\newcommand{\gitaddfile}[1]{\command{git add #1}}

\newcommand{\gitadd}{\command{git add}}
\newcommand{\gitcommit}{\command{git commit}}
\newcommand{\gitpush}{\command{git push}}
\newcommand{\gitpull}{\command{git pull}}

\newcommand{\gitcommitmainprogram}{\command{git commit src/www/index.html -m "Your
descriptive commit message"}}

% Use this when displaying a new command

\newcommand{\command}[1]{``\lstinline{#1}''}
\newcommand{\program}[1]{\lstinline{#1}}
\newcommand{\url}[1]{\lstinline{#1}}
\newcommand{\channel}[1]{\lstinline{#1}}
\newcommand{\option}[1]{``{#1}''}
\newcommand{\step}[1]{``{#1}''}

\usepackage{pifont}
\newcommand{\checkmark}{\ding{51}}
\newcommand{\naughtmark}{\ding{55}}

\usepackage{listings}
\lstset{
  basicstyle=\small\ttfamily,
  columns=flexible,
  breaklines=true
}

\usepackage{fancyhdr}

\usepackage[margin=1in]{geometry}
\usepackage{fancyhdr}

\pagestyle{fancy}

\fancyhf{}
\rhead{Computer Science 103}
\lhead{Laboratory Assignment \assignmentnumber{}}
\rfoot{Page \thepage}
\lfoot{\duedate}

\usepackage{titlesec}
\titlespacing\section{0pt}{6pt plus 4pt minus 2pt}{4pt plus 2pt minus 2pt}

\newcommand{\labtitle}[1]
{
  \begin{center}
    \begin{center}
      \bf
      CMPSC 103\\Web Development\\
      Spring 2018\\
      \medskip
    \end{center}
    \bf
    #1
  \end{center}
}

\begin{document}

\thispagestyle{empty}

\labtitle{Laboratory \assignmentnumber{} \\ \assigneddate{} \\ \duedate{}}

\section*{Objectives}

To continue to practice using GitHub to access the files for a laboratory
assignment. Also, to learn more about using the Ubuntu operating system and
software development programs such as a ``terminal window'' and an ``HTML
linter''. Next, to learn how to write an HTML document that includes headers,
content, and additional tags, specifically implementing an extended version of
the``travel photographs'' example in Figure 3.9 of the textbook. You will extend
your travel web site so that it has more advanced features such as the use
of \program{<blockquote>}, a customized style, and an emoji for each photograph
review. Finally, you will learn more about running a web server and using an
automated grading tool to assess your completion of a web development project.

\section*{Suggestions for Success}

\begin{itemize}
  \setlength{\itemsep}{0pt}

\item {\bf Use the laboratory computers}. The computers in this laboratory feature specialized software for completing
  this course's laboratory and practical assignments. If it is necessary for you to work on a different machine, be sure
  to regularly transfer your work to a laboratory machine so that you can check its correctness. If you cannot use a
  laboratory computer and you need help with the configuration of your own laptop, then please carefully explain its
  setup to a teaching assistant or the course instructor when you are asking questions.

\item {\bf Follow each step carefully}. Slowly read each sentence in the assignment sheet, making sure that you
  precisely follow each instruction. Take notes about each step that you attempt, recording your questions and ideas and
  the challenges that you faced. If you are stuck, then please tell a teaching assistant or instructor what assignment
  step you recently completed.

\item {\bf Regularly ask and answer questions}. Please log into Slack at the start of a laboratory or practical session
  and then join the appropriate channel. If you have a question about one of the steps in an assignment, then you can
  post it to the designated channel. Or, you can ask a student sitting next to you or talk with a teaching assistant or
  the course instructor.

\item {\bf Store your files in GitHub}. Starting with this laboratory assignment, you will be responsible for storing
  all of your files (e.g., JavaScript code and Markdown-based writing) in a Git repository using GitHub Classroom.
  Please verify that you have saved your source code in your Git repository by using \command{git status} to ensure that
  everything is updated. You can see if your assignment submission meets the established correctness requirements by
  using the provided checking tools on your local computer and in checking the commits in GitHub.

\item {\bf Keep all of your files}. Don't delete your programs, output files, and written reports after you submit them
  through GitHub; you will need them again when you study for the quizzes and examinations and work on the other
  laboratory, practical, and final project assignments.

\item {\bf Back up your files regularly}. All of your files are regularly backed-up to the servers in the Department of
  Computer Science and, if you commit your files regularly, stored on GitHub. However, you may want to use a flash
  drive, Google Drive, or your favorite backup method to keep an extra copy of your files on reserve. In the event of
  any type of system failure, you are responsible for ensuring that you have access to a recent backup copy of all your
  files.

\item {\bf Explore teamwork and technologies}. While certain aspects of the laboratory assignments will be challenging
  for you, each part is designed to give you the opportunity to learn both fundamental concepts in the field of computer
  science and explore advanced technologies that are commonly employed at a wide variety of companies. To explore and
  develop new ideas, you should regularly communicate with your team and/or the teaching assistants and tutors.

\item {\bf Hone your technical writing skills}. Computer science assignments require to you write technical
  documentation and descriptions of your experiences when completing each task. Take extra care to ensure that your
  writing is interesting and both grammatically and technically correct, remembering that computer scientists must
  effectively communicate and collaborate with their team members and the tutors, teaching assistants, and course
  instructor.

\item {\bf Review the Honor Code on the syllabus}. While you may discuss your assignments with others, copying source
  code or writing is a violation of Allegheny College's Honor Code.

\end{itemize}

\section*{Reading Assignment}

If you have not done so already, please read all of the relevant ``GitHub
Guides'', available at \url{https://guides.github.com/}, that explain how to use
many of the features that GitHub provides. In particular, make sure that you
have read guides such as ``Mastering Markdown'' and ``Documenting Your Projects
on GitHub''; each of them will help you to understand how to use both GitHub and
GitHub Classroom. To do well on this assignment, you should also read Chapters 1
and 3 in the course textbook, paying close attention to Section 3.5, Figure 3.9
and 3.16, and Table 3.1. Please see the course instructor or a teaching
assistant if you have questions these reading assignments.

\section*{Accessing the Laboratory Assignment on GitHub}

To access the laboratory assignment, you should go into the
\channel{\#announcements} channel in our Slack team and find the announcement
that provides a link for it. Copy this link and paste it into your web browser.
Now, you should accept the laboratory assignment and see that GitHub Classroom
created a new GitHub repository for you to access the assignment's starting
materials and to store the completed version of your assignment. Specifically,
to access your new GitHub repository for this assignment, please click the green
``Accept'' button and then click the link that is prefaced with the label ``Your
assignment has been created here''. If you accepted the assignment and correctly
followed these steps, you should have created a repository with a name like
``Allegheny-Computer-Science-103-Spring-2018/computer-science-103-spring-2018-lab-4-gkapfham''.
Unless you provide the course instructor with documentation of the extenuating
circumstances that you are facing, not accepting the assignment means that you
automatically receive a failing grade for it.

Before you move to the next step of this assignment, please make sure that you
read all of the content on the web site for your new GitHub repository, paying
close attention to the technical details about the commands that you will type
and the output that your program must produce. Now you are ready to download the
starting materials to your laboratory computer. Click the ``Clone or download''
button and, after ensuring that you have selected ``Clone with SSH'', please
copy this command to your clipboard. To enter into your course directory you
should now type \command{cd cs103S2018}. Now, by typing \command{git clone} in
your terminal and then pasting in the string that you copied from the GitHub
site you will download all of the code for this assignment. For instance, if the
course instructor ran the \command{git clone} command in the terminal, it would
look like:

\begin{lstlisting}
  git clone git@github.com:Allegheny-Computer-Science-103-S2018/computer-science-103-spring-2018-lab-4-gkapfham.git
\end{lstlisting}

After this command finishes, you can use \command{cd} to change into the new
directory. If you want to \step{go back} one directory from your current
location, then you can type the command \command{cd ..}. Please continue to use
the \command{cd} and \command{ls} commands to explore the files that you
automatically downloaded from GitHub. What files and directories do you see?
What do you think is their purpose? Please note that this assignment includes an
HTML file with starting tags and a basic template. Spend some time exploring
this file, sharing your discoveries with a \mbox{teaching assistant}.

\section*{Creating an Enhanced Web Site with HTML and CSS}

This laboratory assignment invites you to implement a web site using the HTML
programming language. Specifically, you will create an extended version of the
travel photographs web site that looks like the one in Figure 3.9. Please refer
to your GitHub repository for a screenshot that shows what you web site should
look like when completed. Specifically, please note that this assignment asks
you to complete a web site that is more sophisticated than the one that you
implemented for the last laboratory assignment. To start, you will need to
include the cascading style sheet (CSS) that controls, for instance, the fonts
on the web site. You can achieve this step by adding the code \program{<link
href="css/github.css" rel="stylesheet">} to the \program{<head>} region of the
\mainprogramsource{} file. Note that you also need to have another line of
source code to load the \program{emoji.css} file and its features for including
an emoji in a web site. How would you write that line of HTML source code? You
can refer to Figure~\ref{fig:css} for an example of how to include the two
required CSS files in your \mainprogramsource{} file. Please see the course
instructor if you have any questions about this step.

At this point, you are ready to add, in the \program{<body>} region of the
\mainprogram{} file, the code needed to display the provided image that is in
the \program{img/} directory. Can you find the source code location where this
file name and directory must be included? Using the textbook's content in Figure
3.16 and Table 3.1, can you write the correct source code that includes the
image? Now, you are ready to add the five reviews of the photograph.
Figure~\ref{fig:code} gives an example of the HTML source code that will create
the first review. You should include a customized version of this source code in
your own web site, changing the name, time, and emoji to appropriate values. To
learn more about each emoji that is available for inclusion in your web site,
please visit \url{https://afeld.github.io/emoji-css/}. Ultimately, your web site
should contain at least five photograph reviews that follow the format given in
Figure~\ref{fig:code}. Finally, students should check the correctness of their
HTML by running the command \command{htmlhint src/www/index.html} in their
terminal window. If the web site displays incorrectly, please discuss the
problems with a teaching assistant or the course instructor.

\begin{figure}[t]
  \centering
  \begin{verbatim}
  <head>
    <meta charset="utf-8"/>
    <meta name="viewport" content="width=device-width"/>
    <link href="css/github.css" rel="stylesheet">
    <link href="css/emoji.css" rel="stylesheet">
    <title>Share Your Travels</title>
  </head>
  \end{verbatim}
  \vspace*{-.35in}
  \caption{The Source Code for Loading the CSS in the \mainprogram{} File.}~\label{fig:css}
\end{figure}

\begin{figure}[t]
  \centering
  \begin{verbatim}
    <blockquote>
      <p><b>By Ricardo on <time>February 8, 2018</time></b></p>
      <p>Wow, that is a great photograph. How did you take it?</p>
      <p>I would describe this photograph as <i class="em em---1"></i> </p>
    </blockquote>
  \end{verbatim}
  \vspace*{-.35in}
  \caption{The Source Code for Including a Review in the \mainprogram{} File.}~\label{fig:code}
  \vspace*{-.25in}
\end{figure}

Now, you can run your web server by typing the command \command{serve src/www
4250}. At this point, you can start your web browser and go to the site
\url{http://localhost:4250/}. Please notice that when you access the web site in
this fashion, you do not need to specify the specific HTML file. Why is this the
case? Once you understand the answer to this question, please check to see if
your new web site looks correct. If it is not, then continue to edit and check
it until the files are correct. You should also ensure that, when you run the
web server, it produces output that is similar to (but perhaps not exactly the
same as) that which is available in Figure~\ref{fig:output}. Do you see that
your web server is returning two CSS files and an image? If not, then please
make sure that you revise your HTML source code and/or check the configuration
of your web server.

As you iteratively complete this assignment, you should regularly commit files
to your GitHub repository, using the ``Git Cheat Sheet'' and following the steps
that are described in the next section. Also, note that your web server will
require a dedicated terminal when it is running. After completing the
assignment, reflect on the entire process. What step did you find to be the most
challenging? You should write your reflections in a file, called \reflection{},
that also uses the Markdown language. To complete this aspect of the assignment,
you should write one high-quality paragraph that reports on your experiences.
Now, verbally share your experiences with another class member and the
instructor and at least one the teaching assistants! Finally, please take the
time to answer the other questions in the \reflection{} file. For instance, make
sure that you understand and can explain how the HTML and CSS files work
together to format the text and enable the display of an emoji in the fives
reviews given below the photograph.

\section*{Checking the Correctness of Your Web Site and Writing}

\begin{figure}[t]
  \centering
  \begin{verbatim}
  Thin 1.7.2 available at http://0.0.0.0:4250
  Thin web server (v1.7.2 codename Bachmanity)
  Maximum connections set to 1024
  Listening on 0.0.0.0:4250, CTRL+C to stop
  127.0.0.1 - - [15/Feb/2018 10:48:10] "GET / HTTP/1.1" 200 - 0.0021
  127.0.0.1 - - [15/Feb/2018 10:48:11] "GET /css/github.css HTTP/1.1" 200 11629 0.0006
  127.0.0.1 - - [15/Feb/2018 10:48:11] "GET /css/emoji.css HTTP/1.1" 200 390578 0.0012
  127.0.0.1 - - [15/Feb/2018 10:48:11] "GET /img/plane.jpg HTTP/1.1" 200 22125 0.0005
  \end{verbatim}
  \vspace*{-.35in}
  \caption{The Output From Running a Web Server.}~\label{fig:output}
  \vspace*{-.25in}
\end{figure}

The Markdown file that contains your reflection must adhere to the standards
described in the Markdown Syntax Guide
\url{https://guides.github.com/features/mastering-markdown/}. Finally, your
\reflection{} file should adhere to the Markdown standards established by the
\step{Markdown linting} tool available at
\url{https://github.com/markdownlint/markdownlint/} and the writing standards
set by the \step{prose linting} tool from \url{http://proselint.com/}. Instead
of requiring you to manually check that your deliverables adhere to these
industry-accepted standards, the GatorGrader tool that you will use in this
laboratory assignment makes it easy for you to automatically check if your
submission meets the standards for correctness. For instance, GatorGrader will
check to ensure that \mainprogram{} has the required sections and photograph
reviews.

To get started with the use of GatorGrader, type the command \gatorgraderstart{}
in your terminal window. Once this step completes you can type
\gatorgradercheck{}. If your work does not meet all of the assignment's
requirements, then you will see the following output in your terminal:
\command{Overall, are there any mistakes in the assignment? Yes}. If you do have
mistakes in your assignment, then you will need to review GatorGrader's output,
find the mistake, and try to fix it. Once your static web site displays
correctly, fulfilling at least some of the assignment's requirements, you should
transfer your files to GitHub using the \gitcommit{} and \gitpush{} commands.
For example, if you want to signal that the \mainprogramsource{} file has been
changed and is ready for transfer to GitHub you would first type
\gitcommitmainprogram{} in your terminal, followed by typing \gitpush{} and
checking to see that the transfer to GitHub is successful. If you notice that
transferring your files to GitHub did not work, then please read the terminal
messages and try to determine why.

After the course instructor enables \step{continuous integration} with a system
called Travis CI, when you use the \gitpush{} command to transfer your source
code to your GitHub repository, Travis CI will initialize a \step{build} of your
assignment, checking to see if it meets all of the requirements. If both your
source code and writing meet all of the established requirements, then you will
see a green \checkmark{} in the listing of commits in GitHub after awhile. If
your submission does not meet the requirements, a red \naughtmark{} will appear
instead. The instructor will reduce a student's grade for this assignment if the
red \naughtmark{} appears on the last commit in GitHub immediately before the
assignment's due date. Yet, if the green \checkmark{} appears on the last commit
in your GitHub repository, then you satisfied all of the main checks, thereby
allowing the course instructor to evaluate other aspects of your source code and
writing, as further described in the \step{Evaluation} section of this
assignment sheet. Unless you provide the instructor with documentation of the
extenuating circumstances that you are facing, no late work will be considered
towards your grade for this laboratory assignment.

\section*{Summary of the Required Deliverables}

\noindent Students do not need to submit printed source code or technical writing for any assignment in this course.
Instead, this assignment invites you to submit, using GitHub, the following deliverables.

\begin{enumerate}

\setlength{\itemsep}{0in}

\item Stored in \reflection{}, a one-paragraph answer to all of the stated
  questions. For every challenge that you encountered, please explain your
  solution for it.

\item A properly formatted and correct version of \mainprogramsource{} that both
  meets all of the established requirements and contains the correct HTML and
  the desired static web site.

\end{enumerate}

\section*{Evaluation of Your Laboratory Assignment}

Using a report that the instructor shares with you through the commit log in GitHub, you will privately received a grade
on this assignment and feedback on your submitted deliverables. Your grade for the assignment will be a function of the
whether or not it was submitted in a timely fashion and if your program received a green \checkmark{} indicating that it
met all of the requirements. Other factors will also influence your final grade on the assignment. In addition to
studying the efficiency and effectiveness of your Markdown, the instructor will also evaluate the accuracy of both your
technical writing and the contents of your source code. If your submission receives a red \naughtmark{}, the instructor
will reduce your grade for the assignment while still considering the regularity with which you committed to your GitHub
repository and the overall quality of your partially completed work. Please see the instructor if you have questions
about the evaluation of this laboratory assignment.

\section*{Adhering to the Honor Code}

In adherence to the Honor Code, students should complete this assignment on an individual basis. While it is appropriate
for students in this class to have high-level conversations about the assignment, it is necessary to distinguish
carefully between the student who discusses the principles underlying a problem with others and the student who produces
assignments that are identical to, or merely variations on, someone else's work. Deliverables (e.g., HTML source code or
Markdown-based technical writing) that are nearly identical to the work of others will be taken as evidence of violating
the \mbox{Honor Code}. Please see the course instructor if you have questions about this policy.

\end{document}
