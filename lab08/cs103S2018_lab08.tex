\documentclass[11pt]{article}

% NOTE: The "Edit" sections are changed for each assignment

% Edit these commands for each assignment

\newcommand{\assignmentduedate}{April 3}
\newcommand{\assignmentassignedate}{March 29}
\newcommand{\assignmentnumber}{Eight}

\newcommand{\labyear}{2018}
\newcommand{\labday}{Thursday}
\newcommand{\labtime}{2:30 pm}

\newcommand{\assigneddate}{Assigned: \labday, \assignmentassignedate, \labyear{} at \labtime{}}
\newcommand{\duedate}{Due: \labday, \assignmentduedate, \labyear{} at \labtime{}}

% Edit these commands to give the name to the main program

\newcommand{\mainprogram}{\lstinline{index.html}}
\newcommand{\mainprogramsource}{\lstinline{src/www/index.html}}

\newcommand{\secondprogram}{\lstinline{site.css}}
\newcommand{\secondprogramsource}{\lstinline{src/www/css/site.css}}

% Edit this commands to describe key deliverables

\newcommand{\reflection}{\lstinline{writing/reflection.md}}
\newcommand{\screenshot}{\lstinline{images/screenshot.png}}

% Commands to describe key development tasks

% --> Running gatorgrader.sh
\newcommand{\gatorgraderstart}{\command{./gatorgrader.sh --start}}
\newcommand{\gatorgradercheck}{\command{./gatorgrader.sh --check}}

% Commands to describe key git tasks

% NOTE: Could be improved, problems due to nesting

\newcommand{\gitcommitfile}[1]{\command{git commit #1}}
\newcommand{\gitaddfile}[1]{\command{git add #1}}

\newcommand{\gitadd}{\command{git add}}
\newcommand{\gitcommit}{\command{git commit}}
\newcommand{\gitpush}{\command{git push}}
\newcommand{\gitpull}{\command{git pull}}

\newcommand{\gitcommitmainprogram}{\command{git commit src/www/index.html -m "Your
descriptive commit message"}}

% Use this when displaying a new command

\newcommand{\command}[1]{``\lstinline{#1}''}
\newcommand{\program}[1]{\lstinline{#1}}
\newcommand{\url}[1]{\lstinline{#1}}
\newcommand{\channel}[1]{\lstinline{#1}}
\newcommand{\option}[1]{``{#1}''}
\newcommand{\step}[1]{``{#1}''}

\usepackage{pifont}
\newcommand{\checkmark}{\ding{51}}
\newcommand{\naughtmark}{\ding{55}}

\usepackage{listings}
\lstset{
  basicstyle=\small\ttfamily,
  columns=flexible,
  breaklines=true
}

\usepackage{fancyhdr}

\usepackage[margin=1in]{geometry}
\usepackage{fancyhdr}

\pagestyle{fancy}

\fancyhf{}
\rhead{Computer Science 103}
\lhead{Laboratory Assignment \assignmentnumber{}}
\rfoot{Page \thepage}
\lfoot{\duedate}

\usepackage{titlesec}
\titlespacing\section{0pt}{6pt plus 4pt minus 2pt}{4pt plus 2pt minus 2pt}

\newcommand{\labtitle}[1]
{
  \begin{center}
    \begin{center}
      \bf
      CMPSC 103\\Web Development\\
      Spring 2018\\
      \medskip
    \end{center}
    \bf
    #1
  \end{center}
}

\begin{document}

\thispagestyle{empty}

\labtitle{Laboratory \assignmentnumber{} \\ \assigneddate{} \\ \duedate{}}

\section*{Objectives}

As an investigation of the fundamentals of web media and a review of your
previous work with the HTML and CSS languages, you will implement and test a
comprehensive image gallery similar to the one that the book provides in Figure
6.40. Specifically, you will learn how to use command-line image conversion and
manipulation tools to create images that have different dimensions. You will
then arrange these images into a gallery with a tabular format that is styled by
appropriate CSS rules. While it is acceptable to use HTML tables to create your
web site, students may optionally investigate the use of CSS layout rules, as
described in Chapter 7, to design a flexible and responsive layout for their
gallery of images. Finally, you will continue to practice running a web server
and using software tools that support the implementation, testing, and debugging
of a web site.

\section*{Suggestions for Success}

\begin{itemize}
  \setlength{\itemsep}{0pt}

\item {\bf Use the laboratory computers}. The computers in this laboratory feature specialized software for completing
  this course's laboratory and practical assignments. If it is necessary for you to work on a different machine, be sure
  to regularly transfer your work to a laboratory machine so that you can check its correctness. If you cannot use a
  laboratory computer and you need help with the configuration of your own laptop, then please carefully explain its
  setup to a teaching assistant or the course instructor when you are asking questions.

\item {\bf Follow each step carefully}. Slowly read each sentence in the assignment sheet, making sure that you
  precisely follow each instruction. Take notes about each step that you attempt, recording your questions and ideas and
  the challenges that you faced. If you are stuck, then please tell a teaching assistant or instructor what assignment
  step you recently completed.

\item {\bf Regularly ask and answer questions}. Please log into Slack at the start of a laboratory or practical session
  and then join the appropriate channel. If you have a question about one of the steps in an assignment, then you can
  post it to the designated channel. Or, you can ask a student sitting next to you or talk with a teaching assistant or
  the course instructor.

\item {\bf Store your files in GitHub}. Starting with this laboratory assignment, you will be responsible for storing
  all of your files (e.g., HTML and CSS code and Markdown-based writing) in a Git repository using GitHub Classroom.
  Please verify that you have saved your source code in your Git repository by using \command{git status} to ensure that
  everything is updated. You can see if your assignment submission meets the established correctness requirements by
  using the provided checking tools on your local computer and in checking the commits in GitHub.

\item {\bf Keep all of your files}. Don't delete your programs, output files, and written reports after you submit them
  through GitHub; you will need them again when you study for the quizzes and examinations and work on the other
  laboratory, practical, and final project assignments.

\item {\bf Back up your files regularly}. All of your files are regularly backed-up to the servers in the Department of
  Computer Science and, if you commit your files regularly, stored on GitHub. However, you may want to use a flash
  drive, Google Drive, or your favorite backup method to keep an extra copy of your files on reserve. In the event of
  any type of system failure, you are responsible for ensuring that you have access to a recent backup copy of all your
  files.

\item {\bf Explore teamwork and technologies}. While certain aspects of the laboratory assignments will be challenging
  for you, each part is designed to give you the opportunity to learn both fundamental concepts in the field of computer
  science and explore advanced technologies that are commonly employed at a wide variety of companies. To explore and
  develop new ideas, you should regularly communicate with your team and/or the teaching assistants and tutors.

\item {\bf Hone your technical writing skills}. Computer science assignments require to you write technical
  documentation and descriptions of your experiences when completing each task. Take extra care to ensure that your
  writing is interesting and both grammatically and technically correct, remembering that computer scientists must
  effectively communicate and collaborate with their team members and the tutors, teaching assistants, and course
  instructor.

  % \item {\bf Review the Honor Code on the syllabus}. While you may discuss your
  % assignments with others, copying source code or writing is a violation of
  % Allegheny College's Honor Code.

\end{itemize}

\section*{Reading Assignment}

If you have not done so already, please read all of the relevant ``GitHub
Guides'', available at \url{https://guides.github.com/}, that explain how to use
many of the features that GitHub provides. In particular, make sure that you
have read guides such as ``Mastering Markdown'' and ``Documenting Your Projects
on GitHub''; each of them will help you to understand how to use GitHub. To do
well on this assignment, you should also review Chapters 1 and 5 in the course
textbook, further studying the content on web media in Sections 6.1 through 6.4
and Figure 6.40. Please see the course instructor or a teaching assistant if you
have questions about these reading assignments.

\section*{Accessing the Laboratory Assignment on GitHub}

To access the laboratory assignment, you should go into the
\channel{\#announcements} channel in our Slack team and find the announcement
that provides a link for it. Copy this link and paste it into your web browser.
Now, you should accept the laboratory assignment and see that GitHub Classroom
created a new GitHub repository for you to access the assignment's starting
materials and to store the completed version of your assignment. Specifically,
to access your new GitHub repository for this assignment, please click the green
``Accept'' button and then click the link that is prefaced with the label ``Your
assignment has been created here''. If you accepted the assignment and correctly
followed these steps, you should have created a repository with a name like
``Allegheny-Computer-Science-103-Spring-2018/computer-science-103-spring-2018-lab-8-gkapfham''.
Unless you provide the course instructor with documentation of the extenuating
circumstances that you are facing, not accepting the assignment means that you
automatically receive a failing grade for it.

Before you move to the next step of this assignment, please make sure that you
read all of the content on the web site for your new GitHub repository, paying
close attention to the technical details about the commands that you will type
and the content that must be evident in your web site. Now you are ready to
download the starting materials to your laboratory computer. Click the ``Clone
or download'' button and, after ensuring that you have selected ``Clone with
SSH'', please copy this command to your clipboard. To enter into your course
directory you should now type \command{cd cs103S2018}. Now, by typing
\command{git clone} in your terminal and then pasting in the string that you
copied from the GitHub site you will download all of the code for this
assignment. For instance, if the course instructor ran the \command{git clone}
command in the terminal, it would look like:

\begin{lstlisting}
  git clone git@github.com:Allegheny-Computer-Science-103-S2018/computer-science-103-spring-2018-lab-8-gkapfham.git
\end{lstlisting}

After this command finishes, you can use \command{cd} to change into the new
directory. If you want to \step{go back} one directory from your current
location, then you can type the command \command{cd ..}. Please continue to use
the \command{cd} and \command{ls} commands to explore the files that you
automatically downloaded from GitHub. What files and directories do (or, don't)
you see? What do you think is their purpose? Please note that this laboratory
assignment does not include any HTML, CSS, or media files. This is due to the
fact that a key learning objective for this project is to ensure that you can
successfully create your own bespoke web site using existing HTML and CSS source
code.

\section*{Creating and Testing an Image Gallery}

\begin{figure}[t]
  \centering

  % FIRST TWO ROWS

  \begin{tabular}{|c|c|c|}
    \hline
    \begin{minipage}{1.75in}
      \centering
      \vspace*{.1in}
      Small Image Called $I_1$ \\
      Downloaded from Flickr \\
      Display Resolution \\
      Display Image Size
      \vspace*{.1in}
    \end{minipage} &
    \begin{minipage}{1.75in}
      \centering
      \vspace*{.1in}
      Size-Doubled Version of $I_1$ \\
      Resized with \program{convert} \\
      Display Resolution \\
      Display Image Size
      \vspace*{.1in}
    \end{minipage} &
    \begin{minipage}{1.75in}
      \centering
      \vspace*{.1in}
      Size-Tripled Version of $I_1$ \\
      Resized with \program{convert} \\
      Display Resolution \\
      Display Image Size
      \vspace*{.1in}
    \end{minipage} \\
    \hline
  \end{tabular}

  \vspace*{.2in}

  \begin{tabular}{|c|c|c|}
    \hline
    \begin{minipage}{2in}
      \centering
      \vspace*{.1in}
      Small Image Called $I_1$ \\
      Downloaded from Flickr \\
      Display Resolution \\
      Display Image Size
      \vspace*{.1in}
    \end{minipage} &
    \begin{minipage}{2in}
      \centering
      \vspace*{.1in}
      Size-Doubled Version of $I_1$ \\
      Resized with HTML or CSS \\
      Display Resolution \\
      Display Image Size
      \vspace*{.1in}
    \end{minipage} &
    \begin{minipage}{2in}
      \centering
      \vspace*{.1in}
      Size-Tripled Version of $I_1$ \\
      Resized with HTML or CSS \\
      Display Resolution \\
      Display Image Size
      \vspace*{.1in}
    \end{minipage} \\
    \hline
  \end{tabular}

  \vspace*{.2in}

  % SECOND TWO ROWS

  \begin{tabular}{|c|c|c|}
    \hline
    \begin{minipage}{1.75in}
      \centering
      \vspace*{.1in}
      Small Image Called $I_2$ \\
      Downloaded from Flickr \\
      Display Resolution \\
      Display Image Size
      \vspace*{.1in}
    \end{minipage} &
    \begin{minipage}{1.75in}
      \centering
      \vspace*{.1in}
      Size-Doubled Version of $I_2$ \\
      Resized with \program{convert} \\
      Display Resolution \\
      Display Image Size
      \vspace*{.1in}
    \end{minipage} &
    \begin{minipage}{1.75in}
      \centering
      \vspace*{.1in}
      Size-Tripled Version of $I_2$ \\
      Resized with \program{convert} \\
      Display Resolution \\
      Display Image Size
      \vspace*{.1in}
    \end{minipage} \\
    \hline
  \end{tabular}

  \vspace*{.2in}

  \begin{tabular}{|c|c|c|}
    \hline
    \begin{minipage}{2in}
      \centering
      \vspace*{.1in}
      Small Image Called $I_2$ \\
      Downloaded from Flickr \\
      Display Resolution \\
      Display Image Size
      \vspace*{.1in}
    \end{minipage} &
    \begin{minipage}{2in}
      \centering
      \vspace*{.1in}
      Size-Doubled Version of $I_2$ \\
      Resized with HTML or CSS \\
      Display Resolution \\
      Display Image Size
      \vspace*{.1in}
    \end{minipage} &
    \begin{minipage}{2in}
      \centering
      \vspace*{.1in}
      Size-Tripled Version of $I_2$ \\
      Resized with HTML or CSS \\
      Display Resolution \\
      Display Image Size
      \vspace*{.1in}
    \end{minipage} \\
    \hline
  \end{tabular}

  \caption{{\bf A Conceptual Representation of the Image Gallery}. In this
    example, the notation $I_1$ refers to an image with a ``portrait''
    orientation while $I_2$ is an image with ``landscape'' orientation. The
    second textual row in each box explains how you should resize the image
    while the third row indicates that the image gallery should furnish the
    image ``resolution'' as a height and a width. Similarly, the fourth row
    means that the gallery should give the size of the image in kibibytes
  (K).}~\label{fig:wireframe}

  \vspace*{-.25in}

\end{figure}

Figure~\ref{fig:wireframe} provides a ``wireframe'' (i.e., a conceptual
representation) of the image gallery that you are invited to implement and test
for this laboratory assignment. Since certain aspects of your image gallery are
similar to a project in the textbook, you should also review Figure 6.40 to
learn more about the requirements for your web site. First, you should identify
a small image, denoted $I_1$, that has a ``small'' resolution and is in a
``portrait'' orientation. To create the first row in the image gallery, you
should now use the \program{convert} program to resize $I_1$ so that it is
double and triple its original size. Remember, to learn more about this program,
you can type the command \program{man convert} in your terminal window and read
this ``manual page'' to learn the command-line arguments.

As indicated in Figure~\ref{fig:wireframe} and its caption, each image should be
annotated with text that gives its ``resolution'' (i.e., the height and width of
the image) and its file size in kibibytes. To discover an image's size in
kibibytes, you can type the command \program{ls -lh} in the terminal window and
then provide the name of your image file as an argument. You should also repeat
all of these steps for image $I_2$ that has a ``landscape'' oriented. Finally,
to organize your images into a grid, you can use HTML tables, as you implemented
previously. Alternatively, students who want an extra challenge can use HTML and
CSS source code to create a responsive layout for their grid of images.
Ultimately, each student is responsible for using HTML and CSS to implement
their own approach to arranging the labelled images so that they adhere to the
grid-based structure in Figure~\ref{fig:wireframe}.

Please notice that the second and fourth rows of the image gallery must feature
images that are not resized using the \program{convert} program. Instead, you
should learn how to resize images using HTML and/or CSS source code. Critically,
you should think about the size in kibibytes of these ``resized'' images and
consider the trade-offs associated with adopting one of these two different
mechanisms for image resizing. Please don't forget that the second and fourth
rows should still display, in a textual format, the details about the resolution
and file size for these images. Finally, students should remember to check the
correctness of their HTML by running the command \command{htmlhint
src/www/index.html} in their terminal window. If the image gallery in your web
site is incorrect, then please discuss the problems in your HTML or CSS with a
teaching assistant or the instructor.

Remember, you can run your web server by typing the command \command{serve
src/www 4250}. At this point, you can start your web browser and go to the site
\url{http://localhost:4250/}. Please check to see if your new web site looks
correct. If it is not, then continue to edit and check it until the files are
correct. You should also ensure that, when you run the web server, it produces
output suggesting that it is returning the HTML and CSS files and all of the
locally stored images (i.e., both $I_1$ and $I_2$ and all of the resized
versions of these images). If not, then please make sure that you revise your
HTML and/or CSS source code and/or check the configuration of your web server.
Critically, students should note that, depending on how they organize their
files, they may have to create directories in the ``root'' of their web server
to contain, for instance, the images.

% Please see the instructor if you have questions about the CSS or HTML code.

As you complete this assignment, you should regularly commit files to your
GitHub repository, using the ``Git Cheat Sheet'' and following the steps that
are described in the next section. Also, note that your web server will require
a dedicated terminal when it is running. After completing the assignment,
reflect on the entire process. What step did you find to be the most
challenging? You should write your reflections in a file, called \reflection{},
that also uses Markdown. To complete this part of the assignment, you should
write one high-quality paragraph that reports on your experiences. Now, verbally
share your experiences with another class member and the instructor and at least
one the teaching assistants! Finally, please take the time to answer the other
questions in the \reflection{} file. For instance, make sure that you understand
and can explain how the HTML and CSS files work together to format the site and
enable the display of the (resized) images in a tabular format. You should also
be able to explain how you used the \program{convert} and \program{ls} programs
to correctly produce and understand your resized images.

\section*{Checking the Correctness of Your Web Site and Writing}

The Markdown file that contains your reflection must adhere to the standards
described in the Markdown Syntax Guide
\url{https://guides.github.com/features/mastering-markdown/}. Finally, your
\reflection{} file should adhere to the Markdown standards established by the
\step{Markdown linting} tool available at
\url{https://github.com/markdownlint/markdownlint/} and the writing standards
set by the \step{prose linting} tool from \url{http://proselint.com/}. Instead
of requiring you to manually check that your deliverables adhere to these
industry-accepted standards, the GatorGrader tool that you will use in this
laboratory assignment makes it easy for you to automatically check if your
submission meets the standards for correctness. For instance, GatorGrader will
check to ensure that certain files and directories exist in your repository.
Since this assignment asks you to independently create your source code,
GatorGrader does not check for specific code.

To get started with the use of GatorGrader, type the command \gatorgraderstart{}
in your terminal window. Once this step completes you can type
\gatorgradercheck{}. If your work does not meet all of the assignment's
requirements, then you will see the following output in your terminal:
\command{Overall, are there any mistakes in the assignment? Yes}. If you do have
mistakes in your assignment, then you will need to review GatorGrader's output,
find the mistake, and try to fix it. Once your static web site displays
correctly, fulfilling at least some of the assignment's requirements, you should
transfer your files to GitHub using the \gitcommit{} and \gitpush{} commands.
For example, if you want to signal that the \mainprogramsource{} file has been
changed and is ready for transfer to GitHub you would first type
\gitcommitmainprogram{} in your terminal, followed by typing \gitpush{} and
checking to see that the transfer to GitHub is successful. If you notice that
transferring your files to GitHub did not work, then please read the terminal
messages and try to determine why.

After the course instructor enables \step{continuous integration} with a system
called Travis CI, when you use the \gitpush{} command to transfer your source
code to your GitHub repository, Travis CI will initialize a \step{build} of your
assignment, checking to see if it meets all of the requirements. If both your
source code and writing meet all of the established requirements, then you will
see a green \checkmark{} in the listing of commits in GitHub after awhile. If
your submission does not meet the requirements, a red \naughtmark{} will appear
instead. The instructor will reduce a student's grade for this assignment if the
red \naughtmark{} appears on the last commit in GitHub immediately before the
assignment's due date. Yet, if the green \checkmark{} appears on the last commit
in your GitHub repository, then you satisfied all of the main checks, thereby
allowing the course instructor to evaluate other aspects of your source code and
writing, as further described in the \step{Evaluation} section of this
assignment sheet. Unless you provide the instructor with documentation of the
extenuating circumstances that you are facing, no late work will be considered
towards your grade for this laboratory assignment.

\section*{Summary of the Required Deliverables}

\noindent Students do not need to submit printed source code or technical
writing for any assignment in this course. Instead, this assignment invites you
to submit, using GitHub, the following deliverables. Overall, your submitted web
site's image gallery should match the wireframe provided in
Figure~\ref{fig:wireframe}.

\vspace*{-.1in}

\begin{enumerate}

  \setlength{\itemsep}{0in}

\item Stored in \reflection{}, a one-paragraph answer to all of the stated
  questions. Additionally, for every challenge that you encountered, please
  explain your solution for it.

\item Store in \screenshot{}, a PNG image that give the screenshot of the final
  version of your web site. Note that you can use Ubuntu's ``print screen''
  feature to produce this image.

\item A properly formatted and correct version of \mainprogramsource{} and
  \secondprogramsource{} that both meet all of the established requirements and
  contain the correct HTML and CSS.

\end{enumerate}

\vspace*{-.15in}

\section*{Evaluation of Your Laboratory Assignment}

Using a report that the instructor shares with you through the commit log in
GitHub, you will privately received a grade on this assignment and feedback on
your submitted deliverables. Your grade for the assignment will be a function of
the whether or not it was submitted in a timely fashion and if your program
received a green \checkmark{} indicating that it met all of the requirements.
Other factors will also influence your final grade on the assignment. In
addition to studying the efficiency and effectiveness of your Markdown, the
instructor will also evaluate the accuracy of both your technical writing and
the contents of your source code. If your submission receives a red
\naughtmark{}, the instructor will reduce your grade for the assignment while
still considering the regularity with which you committed to your GitHub
repository and the overall quality of your partially completed work. Please see
the instructor if you have questions about the evaluation of this laboratory
assignment.

% \section*{Adhering to the Honor Code}

% In adherence to the Honor Code, students should complete this assignment on an
% individual basis. While it is appropriate for students in this class to have
% high-level conversations about the assignment, it is necessary to distinguish
% carefully between the student who discusses the principles underlying a problem
% with others and the student who produces assignments that are identical to, or
% merely variations on, someone else's work. Deliverables (e.g., HTML source code
% or Markdown-based technical writing) that are nearly identical to the work of
% others will be taken as evidence of violating the \mbox{Honor Code}. Please see
% the course instructor if you have questions about this policy.

\end{document}
